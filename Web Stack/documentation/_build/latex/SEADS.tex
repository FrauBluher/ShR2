% Generated by Sphinx.
\def\sphinxdocclass{report}
\documentclass[letterpaper,10pt,english]{sphinxmanual}
\usepackage[utf8]{inputenc}
\DeclareUnicodeCharacter{00A0}{\nobreakspace}
\usepackage{cmap}
\usepackage[T1]{fontenc}
\usepackage{babel}
\usepackage{times}
\usepackage[Bjarne]{fncychap}
\usepackage{longtable}
\usepackage{sphinx}
\usepackage{multirow}

\addto\captionsenglish{\renewcommand{\figurename}{Fig. }}
\addto\captionsenglish{\renewcommand{\tablename}{Table }}
\floatname{literal-block}{Listing }



\title{SEADS Documentation}
\date{June 02, 2015}
\release{1.0}
\author{Bryan Smith}
\newcommand{\sphinxlogo}{}
\renewcommand{\releasename}{Release}
\makeindex

\makeatletter
\def\PYG@reset{\let\PYG@it=\relax \let\PYG@bf=\relax%
    \let\PYG@ul=\relax \let\PYG@tc=\relax%
    \let\PYG@bc=\relax \let\PYG@ff=\relax}
\def\PYG@tok#1{\csname PYG@tok@#1\endcsname}
\def\PYG@toks#1+{\ifx\relax#1\empty\else%
    \PYG@tok{#1}\expandafter\PYG@toks\fi}
\def\PYG@do#1{\PYG@bc{\PYG@tc{\PYG@ul{%
    \PYG@it{\PYG@bf{\PYG@ff{#1}}}}}}}
\def\PYG#1#2{\PYG@reset\PYG@toks#1+\relax+\PYG@do{#2}}

\expandafter\def\csname PYG@tok@gd\endcsname{\def\PYG@tc##1{\textcolor[rgb]{0.63,0.00,0.00}{##1}}}
\expandafter\def\csname PYG@tok@gu\endcsname{\let\PYG@bf=\textbf\def\PYG@tc##1{\textcolor[rgb]{0.50,0.00,0.50}{##1}}}
\expandafter\def\csname PYG@tok@gt\endcsname{\def\PYG@tc##1{\textcolor[rgb]{0.00,0.27,0.87}{##1}}}
\expandafter\def\csname PYG@tok@gs\endcsname{\let\PYG@bf=\textbf}
\expandafter\def\csname PYG@tok@gr\endcsname{\def\PYG@tc##1{\textcolor[rgb]{1.00,0.00,0.00}{##1}}}
\expandafter\def\csname PYG@tok@cm\endcsname{\let\PYG@it=\textit\def\PYG@tc##1{\textcolor[rgb]{0.25,0.50,0.56}{##1}}}
\expandafter\def\csname PYG@tok@vg\endcsname{\def\PYG@tc##1{\textcolor[rgb]{0.73,0.38,0.84}{##1}}}
\expandafter\def\csname PYG@tok@m\endcsname{\def\PYG@tc##1{\textcolor[rgb]{0.13,0.50,0.31}{##1}}}
\expandafter\def\csname PYG@tok@mh\endcsname{\def\PYG@tc##1{\textcolor[rgb]{0.13,0.50,0.31}{##1}}}
\expandafter\def\csname PYG@tok@cs\endcsname{\def\PYG@tc##1{\textcolor[rgb]{0.25,0.50,0.56}{##1}}\def\PYG@bc##1{\setlength{\fboxsep}{0pt}\colorbox[rgb]{1.00,0.94,0.94}{\strut ##1}}}
\expandafter\def\csname PYG@tok@ge\endcsname{\let\PYG@it=\textit}
\expandafter\def\csname PYG@tok@vc\endcsname{\def\PYG@tc##1{\textcolor[rgb]{0.73,0.38,0.84}{##1}}}
\expandafter\def\csname PYG@tok@il\endcsname{\def\PYG@tc##1{\textcolor[rgb]{0.13,0.50,0.31}{##1}}}
\expandafter\def\csname PYG@tok@go\endcsname{\def\PYG@tc##1{\textcolor[rgb]{0.20,0.20,0.20}{##1}}}
\expandafter\def\csname PYG@tok@cp\endcsname{\def\PYG@tc##1{\textcolor[rgb]{0.00,0.44,0.13}{##1}}}
\expandafter\def\csname PYG@tok@gi\endcsname{\def\PYG@tc##1{\textcolor[rgb]{0.00,0.63,0.00}{##1}}}
\expandafter\def\csname PYG@tok@gh\endcsname{\let\PYG@bf=\textbf\def\PYG@tc##1{\textcolor[rgb]{0.00,0.00,0.50}{##1}}}
\expandafter\def\csname PYG@tok@ni\endcsname{\let\PYG@bf=\textbf\def\PYG@tc##1{\textcolor[rgb]{0.84,0.33,0.22}{##1}}}
\expandafter\def\csname PYG@tok@nl\endcsname{\let\PYG@bf=\textbf\def\PYG@tc##1{\textcolor[rgb]{0.00,0.13,0.44}{##1}}}
\expandafter\def\csname PYG@tok@nn\endcsname{\let\PYG@bf=\textbf\def\PYG@tc##1{\textcolor[rgb]{0.05,0.52,0.71}{##1}}}
\expandafter\def\csname PYG@tok@no\endcsname{\def\PYG@tc##1{\textcolor[rgb]{0.38,0.68,0.84}{##1}}}
\expandafter\def\csname PYG@tok@na\endcsname{\def\PYG@tc##1{\textcolor[rgb]{0.25,0.44,0.63}{##1}}}
\expandafter\def\csname PYG@tok@nb\endcsname{\def\PYG@tc##1{\textcolor[rgb]{0.00,0.44,0.13}{##1}}}
\expandafter\def\csname PYG@tok@nc\endcsname{\let\PYG@bf=\textbf\def\PYG@tc##1{\textcolor[rgb]{0.05,0.52,0.71}{##1}}}
\expandafter\def\csname PYG@tok@nd\endcsname{\let\PYG@bf=\textbf\def\PYG@tc##1{\textcolor[rgb]{0.33,0.33,0.33}{##1}}}
\expandafter\def\csname PYG@tok@ne\endcsname{\def\PYG@tc##1{\textcolor[rgb]{0.00,0.44,0.13}{##1}}}
\expandafter\def\csname PYG@tok@nf\endcsname{\def\PYG@tc##1{\textcolor[rgb]{0.02,0.16,0.49}{##1}}}
\expandafter\def\csname PYG@tok@si\endcsname{\let\PYG@it=\textit\def\PYG@tc##1{\textcolor[rgb]{0.44,0.63,0.82}{##1}}}
\expandafter\def\csname PYG@tok@s2\endcsname{\def\PYG@tc##1{\textcolor[rgb]{0.25,0.44,0.63}{##1}}}
\expandafter\def\csname PYG@tok@vi\endcsname{\def\PYG@tc##1{\textcolor[rgb]{0.73,0.38,0.84}{##1}}}
\expandafter\def\csname PYG@tok@nt\endcsname{\let\PYG@bf=\textbf\def\PYG@tc##1{\textcolor[rgb]{0.02,0.16,0.45}{##1}}}
\expandafter\def\csname PYG@tok@nv\endcsname{\def\PYG@tc##1{\textcolor[rgb]{0.73,0.38,0.84}{##1}}}
\expandafter\def\csname PYG@tok@s1\endcsname{\def\PYG@tc##1{\textcolor[rgb]{0.25,0.44,0.63}{##1}}}
\expandafter\def\csname PYG@tok@gp\endcsname{\let\PYG@bf=\textbf\def\PYG@tc##1{\textcolor[rgb]{0.78,0.36,0.04}{##1}}}
\expandafter\def\csname PYG@tok@sh\endcsname{\def\PYG@tc##1{\textcolor[rgb]{0.25,0.44,0.63}{##1}}}
\expandafter\def\csname PYG@tok@ow\endcsname{\let\PYG@bf=\textbf\def\PYG@tc##1{\textcolor[rgb]{0.00,0.44,0.13}{##1}}}
\expandafter\def\csname PYG@tok@sx\endcsname{\def\PYG@tc##1{\textcolor[rgb]{0.78,0.36,0.04}{##1}}}
\expandafter\def\csname PYG@tok@bp\endcsname{\def\PYG@tc##1{\textcolor[rgb]{0.00,0.44,0.13}{##1}}}
\expandafter\def\csname PYG@tok@c1\endcsname{\let\PYG@it=\textit\def\PYG@tc##1{\textcolor[rgb]{0.25,0.50,0.56}{##1}}}
\expandafter\def\csname PYG@tok@kc\endcsname{\let\PYG@bf=\textbf\def\PYG@tc##1{\textcolor[rgb]{0.00,0.44,0.13}{##1}}}
\expandafter\def\csname PYG@tok@c\endcsname{\let\PYG@it=\textit\def\PYG@tc##1{\textcolor[rgb]{0.25,0.50,0.56}{##1}}}
\expandafter\def\csname PYG@tok@mf\endcsname{\def\PYG@tc##1{\textcolor[rgb]{0.13,0.50,0.31}{##1}}}
\expandafter\def\csname PYG@tok@err\endcsname{\def\PYG@bc##1{\setlength{\fboxsep}{0pt}\fcolorbox[rgb]{1.00,0.00,0.00}{1,1,1}{\strut ##1}}}
\expandafter\def\csname PYG@tok@mb\endcsname{\def\PYG@tc##1{\textcolor[rgb]{0.13,0.50,0.31}{##1}}}
\expandafter\def\csname PYG@tok@ss\endcsname{\def\PYG@tc##1{\textcolor[rgb]{0.32,0.47,0.09}{##1}}}
\expandafter\def\csname PYG@tok@sr\endcsname{\def\PYG@tc##1{\textcolor[rgb]{0.14,0.33,0.53}{##1}}}
\expandafter\def\csname PYG@tok@mo\endcsname{\def\PYG@tc##1{\textcolor[rgb]{0.13,0.50,0.31}{##1}}}
\expandafter\def\csname PYG@tok@kd\endcsname{\let\PYG@bf=\textbf\def\PYG@tc##1{\textcolor[rgb]{0.00,0.44,0.13}{##1}}}
\expandafter\def\csname PYG@tok@mi\endcsname{\def\PYG@tc##1{\textcolor[rgb]{0.13,0.50,0.31}{##1}}}
\expandafter\def\csname PYG@tok@kn\endcsname{\let\PYG@bf=\textbf\def\PYG@tc##1{\textcolor[rgb]{0.00,0.44,0.13}{##1}}}
\expandafter\def\csname PYG@tok@o\endcsname{\def\PYG@tc##1{\textcolor[rgb]{0.40,0.40,0.40}{##1}}}
\expandafter\def\csname PYG@tok@kr\endcsname{\let\PYG@bf=\textbf\def\PYG@tc##1{\textcolor[rgb]{0.00,0.44,0.13}{##1}}}
\expandafter\def\csname PYG@tok@s\endcsname{\def\PYG@tc##1{\textcolor[rgb]{0.25,0.44,0.63}{##1}}}
\expandafter\def\csname PYG@tok@kp\endcsname{\def\PYG@tc##1{\textcolor[rgb]{0.00,0.44,0.13}{##1}}}
\expandafter\def\csname PYG@tok@w\endcsname{\def\PYG@tc##1{\textcolor[rgb]{0.73,0.73,0.73}{##1}}}
\expandafter\def\csname PYG@tok@kt\endcsname{\def\PYG@tc##1{\textcolor[rgb]{0.56,0.13,0.00}{##1}}}
\expandafter\def\csname PYG@tok@sc\endcsname{\def\PYG@tc##1{\textcolor[rgb]{0.25,0.44,0.63}{##1}}}
\expandafter\def\csname PYG@tok@sb\endcsname{\def\PYG@tc##1{\textcolor[rgb]{0.25,0.44,0.63}{##1}}}
\expandafter\def\csname PYG@tok@k\endcsname{\let\PYG@bf=\textbf\def\PYG@tc##1{\textcolor[rgb]{0.00,0.44,0.13}{##1}}}
\expandafter\def\csname PYG@tok@se\endcsname{\let\PYG@bf=\textbf\def\PYG@tc##1{\textcolor[rgb]{0.25,0.44,0.63}{##1}}}
\expandafter\def\csname PYG@tok@sd\endcsname{\let\PYG@it=\textit\def\PYG@tc##1{\textcolor[rgb]{0.25,0.44,0.63}{##1}}}

\def\PYGZbs{\char`\\}
\def\PYGZus{\char`\_}
\def\PYGZob{\char`\{}
\def\PYGZcb{\char`\}}
\def\PYGZca{\char`\^}
\def\PYGZam{\char`\&}
\def\PYGZlt{\char`\<}
\def\PYGZgt{\char`\>}
\def\PYGZsh{\char`\#}
\def\PYGZpc{\char`\%}
\def\PYGZdl{\char`\$}
\def\PYGZhy{\char`\-}
\def\PYGZsq{\char`\'}
\def\PYGZdq{\char`\"}
\def\PYGZti{\char`\~}
% for compatibility with earlier versions
\def\PYGZat{@}
\def\PYGZlb{[}
\def\PYGZrb{]}
\makeatother

\renewcommand\PYGZsq{\textquotesingle}

\begin{document}

\maketitle
\tableofcontents
\phantomsection\label{index::doc}

\begin{quote}\begin{description}
\item[{Author}] \leavevmode
\href{https://github.com/brabsmit}{Bryan Smith}

\item[{License}] \leavevmode
This document is licensed under a \href{http://creativecommons.org/licenses/by-sa/4.0/}{Creative Commons Attribution-ShareAlike 4.0 International License}.

\end{description}\end{quote}

\begin{notice}{note}{Note:}
Official API documentation is available \href{http://seads.io/docs/}{here}.
\end{notice}


\chapter{Installation}
\label{installation:installation}\label{installation:smart-energy-analysis-and-disaggregation-web-components}\label{installation::doc}\label{installation:id1}

\section{Introduction}
\label{installation:introduction}
The SEADS web infrastructure provieds a simple API to read/write the data gathered by the SEAD Light.In addition, this framework is designed to provide in-house visualization and cost analysis of the energy usage. Built on Django, it is encouraged that this environment be augmented with new and improved applications that can interface with the data.

The current setup has all the necessary infrastructure in place for SEAD Lights to send their data. In addition, a proof-of-concept application has been created to interface with the data in a meaningful way.

There are two ways to deploy this system:
\begin{enumerate}
\item {} 
Scalable framework based in the Amazon Cloud Computing Services (recommended)

\item {} 
Atomic install on a single machine (easy)

\end{enumerate}


\section{Installing Scalable Framework in Amazon Compute Cloud (Ubuntu)}
\label{installation:installing-scalable-framework-in-amazon-compute-cloud-ubuntu}
\begin{notice}{note}{Note:}
This project was developed entirely on a Ubuntu build (14.04.2) so these instructions will be tailored towards that build. However, this framework should install semi-peacefully on any OS that can run python/nginx/uwsgi.

These instructions will assume you know how to interface with the amazon AWS console to create new instances. Refer to the \href{http://docs.aws.amazon.com/AWSEC2/latest/UserGuide/EC2\_GetStarted.html}{Getting Started Guide} for more information.
\end{notice}


\subsection{Basic Server Outline}
\label{installation:basic-server-outline}
For the scalable framework to work correctly, there is a bare minimum of 3 servers that need to be always running. Each server has a unique purpose:
\begin{enumerate}
\item {} 
Influxdb Server - A medium/large instance that houses the data from the SEAD Light. Needs to be large to handle the calculations that go into the fanout queries of the database.

\item {} 
Webapp Stateful Server - A server that houses the Django database that holds state information about the web application, such as user credentials and model relations.

\item {} 
Webapp Stateless Server - A skeleton server that serves as the frontend for users connecting to the web interface. This server is a clone of an image used in the auto scaling group.

\end{enumerate}

The servers are setup with the following hierarchy:

The infrastructure is set up in this way to deal with a scalable load in an intelligent way. If there is little to no traffic to the website/REST API, the servers will spin down to a minimal state. However, if load increases, the infrastructure is designed to automatically spin up new instances of the web application so that no single instance is overloaded.


\subsection{InfluxDB Server Setup}
\label{installation:influxdb-server-setup}
To get started, create a medium/large instance for the InfluxDB database. The operating system is recommended to be Ubuntu, but this is not a requirement. This server will only be interfaced by the stateless web servers under the following circumstances: 1) A user requests device data via the web application, 2) A user/device interacts with the REST API to read/write device data.

The following ports should be opened for the InfluxDB instance:

\begin{tabulary}{\linewidth}{|L|L|L|L|L|}
\hline
\textsf{\relax 
Type
} & \textsf{\relax 
Protocol
} & \textsf{\relax 
Port Range
} & \textsf{\relax 
Source
} & \textsf{\relax 
Purpose
}\\
\hline
SSH
 & 
TCP
 & 
22
 & 
0.0.0.0/0
 & 
Self explanatory
\\
\hline
Custom TCP Rule
 & 
TCP
 & 
8083
 & 
0.0.0.0/0
 & 
Exposes the database web API for interactive use
\\
\hline
Custom TCP Rule
 & 
TCP
 & 
8086
 & 
0.0.0.0/0
 & 
Exposes the database REST port for the python interface
\\
\hline\end{tabulary}


Once the server has booted, connect to it via ssh and do the following:
\begin{enumerate}
\item {} 
Install Git:

\begin{Verbatim}[commandchars=\\\{\}]
sudo apt\PYGZhy{}get install git
\end{Verbatim}

\item {} 
Clone the SEADS repository:

\begin{Verbatim}[commandchars=\\\{\}]
git clone https://github.com/Fraubluher/ShR2/
\end{Verbatim}

\item {} 
Run the deploy script for influxdb:

\begin{Verbatim}[commandchars=\\\{\}]
cd ShR2/Web\PYGZbs{} Stack/
sudo ./deploy\PYGZus{}database.sh
\end{Verbatim}

\item {} 
Reboot the server:

\begin{Verbatim}[commandchars=\\\{\}]
sudo reboot
\end{Verbatim}

\item {} 
Configure the database:

\begin{Verbatim}[commandchars=\\\{\}]
curl \PYGZhy{}X POST \PYGZsq{}http://localhost:8086/db?u=\PYGZlt{}username\PYGZgt{}\PYGZam{}p=\PYGZlt{}password\PYGZgt{}\PYGZsq{} \PYGZbs{}
         \PYGZhy{}d \PYGZsq{}\PYGZob{}\PYGZdq{}name\PYGZdq{}: \PYGZdq{}seads\PYGZdq{}\PYGZcb{}\PYGZsq{}
\end{Verbatim}

\end{enumerate}

The InfluxDB server is now ready to respond to requests.


\subsection{Django Web Application Stateful Server Setup}
\label{installation:django-web-application-stateful-server-setup}
This process will walk you through the process of installing a stateful server for the Django web application.

Create a tiny/small instance running Ubuntu (the operating system is recommended to be Ubuntu, but this is not a requirement).

The following ports should be opened for the stateful server:

\begin{tabulary}{\linewidth}{|L|L|L|L|L|}
\hline
\textsf{\relax 
Type
} & \textsf{\relax 
Protocol
} & \textsf{\relax 
Port Range
} & \textsf{\relax 
Source
} & \textsf{\relax 
Purpose
}\\
\hline
SSH
 & 
TCP
 & 
22
 & 
0.0.0.0/0
 & 
Self explanatory
\\
\hline
MYSQL
 & 
TCP
 & 
3306
 & 
0.0.0.0/0
 & 
Port for remotely interfacing with Django database
\\
\hline\end{tabulary}


Once the server has booted, connect to it via ssh and do the following:
\begin{enumerate}
\item {} 
Install Git:

\begin{Verbatim}[commandchars=\\\{\}]
sudo apt\PYGZhy{}get install git
\end{Verbatim}

\item {} 
Clone the SEADS repository:

\begin{Verbatim}[commandchars=\\\{\}]
git clone https://github.com/Fraubluher/ShR2/
\end{Verbatim}

\item {} 
Run the deploy script for influxdb:

\begin{Verbatim}[commandchars=\\\{\}]
cd ShR2/Web\PYGZbs{} Stack/
sudo ./deploy\PYGZus{}webapp\PYGZus{}stateful.sh
\end{Verbatim}

\end{enumerate}

This script will take you through the process of creating the MySQL database to be used by the stateless servers in the future. You will be prompted to create a root user on the database, remember the credentials for later.

This script will install all the necessary dependencies for the Django project. This will take a while, grab a beverage.

Near the end, several prompts will appear. You will be prompted to create the Django user in the MySQL database that is used to interface with the stateless servers. Leaving prompts blank will roll over to their default values indicated in the parentheses.
\begin{enumerate}
\setcounter{enumi}{3}
\item {} 
Reboot the server:

\begin{Verbatim}[commandchars=\\\{\}]
sudo reboot
\end{Verbatim}

\end{enumerate}

This server should now be properly configured to run as a stateful implementation of the web application.


\subsection{Django Web Application Stateless Server Setup}
\label{installation:django-web-application-stateless-server-setup}
The final step in assembling the server infrastructure is to create a stateless instance of the web application. This will provide the basis for which an auto scaler can instantiate more/less instances of the web application automatically.

Create a tiny/small instance running Ubuntu (the operating system is recommended to be Ubuntu, but this is not a requirement).

The following ports should be opened for the stateful server:

\begin{tabulary}{\linewidth}{|L|L|L|L|L|}
\hline
\textsf{\relax 
Type
} & \textsf{\relax 
Protocol
} & \textsf{\relax 
Port Range
} & \textsf{\relax 
Source
} & \textsf{\relax 
Purpose
}\\
\hline
SSH
 & 
TCP
 & 
22
 & 
0.0.0.0/0
 & 
Self explanatory
\\
\hline
HTTP
 & 
TCP
 & 
80
 & 
0.0.0.0/0
 & 
Self explanatory
\\
\hline\end{tabulary}


Since this is the forward-facing instance, the HTTP port is opened for clients to connect to. This allows both end users and SEAD Lights to connect and interact.

Once the server has booted, connect to it via ssh and do the following:
\begin{enumerate}
\item {} 
Install Git:

\begin{Verbatim}[commandchars=\\\{\}]
sudo apt\PYGZhy{}get install git
\end{Verbatim}

\item {} 
Clone the SEADS repository:

\begin{Verbatim}[commandchars=\\\{\}]
git clone https://github.com/Fraubluher/ShR2/
\end{Verbatim}

\item {} 
Run the deploy script for influxdb:

\begin{Verbatim}[commandchars=\\\{\}]
cd ShR2/Web\PYGZbs{} Stack/
sudo ./deploy\PYGZus{}webapp\PYGZus{}stateless.sh
\end{Verbatim}

\end{enumerate}

When this script runs, it will prompt for the address for the remote database (Django database host address). This is the address of the server created in the previous step.
\begin{enumerate}
\setcounter{enumi}{3}
\item {} 
Reboot the server:

\begin{Verbatim}[commandchars=\\\{\}]
sudo reboot
\end{Verbatim}

\end{enumerate}

When the server reboots, you should now be able to connect to it from a web browser and test out the functionality. The stateless server is the address in which clients and SEAD Lights should connect.


\subsection{Finishing Up}
\label{installation:finishing-up}
At this point, you have a functioning server framework that is eligible for load balancing and auto scaling. This guide does not get into the specifics since it is unique to the cloud service being used.

In general, these are the steps you should follow:
\begin{enumerate}
\item {} 
Create an image from the fully-configured webapp stateful server.

\item {} 
Configure and auto scaling group based on the image.

\item {} 
Configure a load balancer based off the auto scaling group.

\end{enumerate}

If you choose to link the server's address to a domain name after configuring a load balancer, a CNAME record must be created with the DNS provider with the load balancer's address.


\section{Installing Atomic Server}
\label{installation:installing-atomic-server}
\begin{notice}{note}{Note:}
This project was developed entirely on a Ubuntu build (14.04.2) so these instructions will be tailored towards that build. However, this framework should install semi-peacefully on any OS that can run python/nginx/uwsgi.

These instructions will not focus on deploying in the Amazon Compute Cloud, however it is certainly possible to do so.
\end{notice}


\subsection{Basic Server Outline}
\label{installation:id2}
This server will comprise all aspects of the project on a single machine. This type of setup is intended for a small user base on the order of 10's of users. Any more and you should consider adopting the scalable approach above. It is recommended to use Ubuntu simply because this platform was developed and tested solely on Ubuntu.

To get started, open up the following ports on your machine:

\begin{tabulary}{\linewidth}{|L|L|L|L|L|}
\hline
\textsf{\relax 
Type
} & \textsf{\relax 
Protocol
} & \textsf{\relax 
Port Range
} & \textsf{\relax 
Source
} & \textsf{\relax 
Purpose
}\\
\hline
SSH
 & 
TCP
 & 
22
 & 
0.0.0.0/0
 & 
Self explanatory
\\
\hline
Custom TCP Rule
 & 
TCP
 & 
8083
 & 
0.0.0.0/0
 & 
Exposes the database web API for interactive use
\\
\hline
Custom TCP Rule
 & 
TCP
 & 
8086
 & 
0.0.0.0/0
 & 
Exposes the database REST port for the python interface
\\
\hline
MYSQL
 & 
TCP
 & 
3306
 & 
0.0.0.0/0
 & 
Port for remotely interfacing with Django database
\\
\hline
HTTP
 & 
TCP
 & 
80
 & 
0.0.0.0/0
 & 
Self explanatory
\\
\hline\end{tabulary}

\begin{enumerate}
\item {} 
Install Git:

\begin{Verbatim}[commandchars=\\\{\}]
sudo apt\PYGZhy{}get install git
\end{Verbatim}

\item {} 
Clone the SEADS repository:

\begin{Verbatim}[commandchars=\\\{\}]
git clone https://github.com/Fraubluher/ShR2/
\end{Verbatim}

\item {} 
Run the deploy script for influxdb:

\begin{Verbatim}[commandchars=\\\{\}]
cd ShR2/Web\PYGZbs{} Stack/
sudo ./deploy\PYGZus{}webapp\PYGZus{}stateful.sh
\end{Verbatim}

\end{enumerate}

This script will walk you through creating and configuring the databases needed. For any prompt asking for an address, enter `localhost'.
\begin{enumerate}
\setcounter{enumi}{3}
\item {} 
Reboot the server:

\begin{Verbatim}[commandchars=\\\{\}]
sudo reboot
\end{Verbatim}

\end{enumerate}

When the server reboots, verify it works by visiting the server from a webpage. All basic functionality should now exist.


\chapter{Getting Started}
\label{getting-started:getting-started}\label{getting-started::doc}\label{getting-started:id1}
At this point, the project is now ready to begin accepting clients and devices.

To get devices connected to your project, simply point their database address at the server that faces the internet (for the scalable framework, this is the stateless server or load balancer). This will cause the devices to attempt to find themselves in the server, fail, then register themselves in the system.

From here, users can register to the website and pair to devices via their serial number.


\section{API Documentation}
\label{getting-started:api-documentation}
You can interact with the server's API by navigating to /docs. This interface was designed to allow developers easy access to the framework so that they can learn it rapidly and integrate new types of devices into the database.

The REST API is set up to be very trusting. There are no checks for malicious behavior and we assume all users are benign. It is possible to alter any and all properties of a device such as changing the owner via the API. In the future, it is recommended to include API token authentication to prevent malicious attacks.


\section{Administrative Interface}
\label{getting-started:administrative-interface}
In addition to the front facing web interface, there is an administrative interface for managing database models directly. This interface was designed to allow an administrator the ability to alter the way the website functions without having to interface with the source code directly.

An administrator can use this interface to create/modify devices, circuit types, and appliance directly. This would presumably become useful when the algorithms on the SEAD Light mature to the point where disaggregation by appliance is feasable.

In addition, an administrator can also interface with the facets of the web application, including how event notifications are handled as well as add/modify interval notifications. These are the emails sent to users after a certain event has been detected or an interval has elapsed.

There is the ability to add/modify rate plans, territories, and utility companies to the web application, giving more realistic cost predictions for a user's device.


\chapter{Applications}
\label{applications:applications}\label{applications::doc}\label{applications:id1}
Three distinct applications were created that encompass the SEAD web framework:
\begin{enumerate}
\item {} 
Microdata - Interfaces with the devices

\item {} 
Webapp - Interfaces with the clients

\item {} 
Farmer - Manages the devices

\end{enumerate}

And (optionally) a fourth:
\begin{enumerate}
\setcounter{enumi}{3}
\item {} 
Debug - Models that assist in debugging the project

\end{enumerate}

The applications were designed such that the Webapp application is hot-swappable. Since it is a proof of concept, it can either be added to or replaced altogether. Microdata and Farmer are able to run without Webapp, however their data cannot be interfaced with in a meaningful way without a web application.

\href{http://i.imgur.com/uw3BtWt.png}{Full Size}

What follows is the breakdown of each application and how they work.


\section{Microdata Application}
\label{modules/microdata:microdata-application}\label{modules/microdata::doc}
This application is the direct interface between devices and the InfluxDB database. It is responsible for exposing the REST API endpoints.


\subsection{Subpackages}
\label{modules/microdata:subpackages}

\subsubsection{microdata.management package}
\label{modules/microdata.management:microdata-management-package}\label{modules/microdata.management::doc}

\paragraph{Subpackages}
\label{modules/microdata.management:subpackages}

\subparagraph{microdata.management.commands package}
\label{modules/microdata.management.commands::doc}\label{modules/microdata.management.commands:microdata-management-commands-package}

\subparagraph{Submodules}
\label{modules/microdata.management.commands:submodules}

\subparagraph{microdata.management.commands.archive\_database module}
\label{modules/microdata.management.commands:microdata-management-commands-archive-database-module}\label{modules/microdata.management.commands:module-microdata.management.commands.archive_database}\index{microdata.management.commands.archive\_database (module)}\index{Command (class in microdata.management.commands.archive\_database)}

\begin{fulllineitems}
\phantomsection\label{modules/microdata.management.commands:microdata.management.commands.archive_database.Command}\pysigline{\strong{class }\code{microdata.management.commands.archive\_database.}\bfcode{Command}}
Bases: \code{django.core.management.base.BaseCommand}
\index{args (microdata.management.commands.archive\_database.Command attribute)}

\begin{fulllineitems}
\phantomsection\label{modules/microdata.management.commands:microdata.management.commands.archive_database.Command.args}\pysigline{\bfcode{args}\strong{ = `'}}
\end{fulllineitems}

\index{handle() (microdata.management.commands.archive\_database.Command method)}

\begin{fulllineitems}
\phantomsection\label{modules/microdata.management.commands:microdata.management.commands.archive_database.Command.handle}\pysiglinewithargsret{\bfcode{handle}}{\emph{*args}, \emph{**options}}{}
\end{fulllineitems}

\index{help (microdata.management.commands.archive\_database.Command attribute)}

\begin{fulllineitems}
\phantomsection\label{modules/microdata.management.commands:microdata.management.commands.archive_database.Command.help}\pysigline{\bfcode{help}\strong{ = `Backs up the data for each device relative to its retention policy.'}}
\end{fulllineitems}


\end{fulllineitems}

\index{safe\_list\_get() (in module microdata.management.commands.archive\_database)}

\begin{fulllineitems}
\phantomsection\label{modules/microdata.management.commands:microdata.management.commands.archive_database.safe_list_get}\pysiglinewithargsret{\code{microdata.management.commands.archive\_database.}\bfcode{safe\_list\_get}}{\emph{l}, \emph{idx}, \emph{default}}{}
\end{fulllineitems}



\subparagraph{microdata.management.commands.check\_glacier\_jobs module}
\label{modules/microdata.management.commands:microdata-management-commands-check-glacier-jobs-module}\label{modules/microdata.management.commands:module-microdata.management.commands.check_glacier_jobs}\index{microdata.management.commands.check\_glacier\_jobs (module)}\index{Command (class in microdata.management.commands.check\_glacier\_jobs)}

\begin{fulllineitems}
\phantomsection\label{modules/microdata.management.commands:microdata.management.commands.check_glacier_jobs.Command}\pysigline{\strong{class }\code{microdata.management.commands.check\_glacier\_jobs.}\bfcode{Command}}
Bases: \code{django.core.management.base.BaseCommand}
\index{handle() (microdata.management.commands.check\_glacier\_jobs.Command method)}

\begin{fulllineitems}
\phantomsection\label{modules/microdata.management.commands:microdata.management.commands.check_glacier_jobs.Command.handle}\pysiglinewithargsret{\bfcode{handle}}{\emph{*args}, \emph{**options}}{}
\end{fulllineitems}


\end{fulllineitems}



\subparagraph{Module contents}
\label{modules/microdata.management.commands:module-microdata.management.commands}\label{modules/microdata.management.commands:module-contents}\index{microdata.management.commands (module)}

\paragraph{Module contents}
\label{modules/microdata.management:module-contents}\label{modules/microdata.management:module-microdata.management}\index{microdata.management (module)}

\subsection{Submodules}
\label{modules/microdata:submodules}

\subsection{microdata.admin module}
\label{modules/microdata:microdata-admin-module}\label{modules/microdata:module-microdata.admin}\index{microdata.admin (module)}\index{ApplianceAdmin (class in microdata.admin)}

\begin{fulllineitems}
\phantomsection\label{modules/microdata:microdata.admin.ApplianceAdmin}\pysiglinewithargsret{\strong{class }\code{microdata.admin.}\bfcode{ApplianceAdmin}}{\emph{model}, \emph{admin\_site}}{}
Bases: \code{django.contrib.admin.options.ModelAdmin}
\index{list\_display (microdata.admin.ApplianceAdmin attribute)}

\begin{fulllineitems}
\phantomsection\label{modules/microdata:microdata.admin.ApplianceAdmin.list_display}\pysigline{\bfcode{list\_display}\strong{ = (`name', `pk', `serial', `chart\_color')}}
\end{fulllineitems}

\index{media (microdata.admin.ApplianceAdmin attribute)}

\begin{fulllineitems}
\phantomsection\label{modules/microdata:microdata.admin.ApplianceAdmin.media}\pysigline{\bfcode{media}}
\end{fulllineitems}


\end{fulllineitems}

\index{CircuitAdmin (class in microdata.admin)}

\begin{fulllineitems}
\phantomsection\label{modules/microdata:microdata.admin.CircuitAdmin}\pysiglinewithargsret{\strong{class }\code{microdata.admin.}\bfcode{CircuitAdmin}}{\emph{model}, \emph{admin\_site}}{}
Bases: \code{django.contrib.admin.options.ModelAdmin}
\index{list\_display (microdata.admin.CircuitAdmin attribute)}

\begin{fulllineitems}
\phantomsection\label{modules/microdata:microdata.admin.CircuitAdmin.list_display}\pysigline{\bfcode{list\_display}\strong{ = (`name', `circuittype', `pk')}}
\end{fulllineitems}

\index{media (microdata.admin.CircuitAdmin attribute)}

\begin{fulllineitems}
\phantomsection\label{modules/microdata:microdata.admin.CircuitAdmin.media}\pysigline{\bfcode{media}}
\end{fulllineitems}


\end{fulllineitems}

\index{CircuitTypeAdmin (class in microdata.admin)}

\begin{fulllineitems}
\phantomsection\label{modules/microdata:microdata.admin.CircuitTypeAdmin}\pysiglinewithargsret{\strong{class }\code{microdata.admin.}\bfcode{CircuitTypeAdmin}}{\emph{model}, \emph{admin\_site}}{}
Bases: \code{django.contrib.admin.options.ModelAdmin}
\index{list\_display (microdata.admin.CircuitTypeAdmin attribute)}

\begin{fulllineitems}
\phantomsection\label{modules/microdata:microdata.admin.CircuitTypeAdmin.list_display}\pysigline{\bfcode{list\_display}\strong{ = (`name', `pk')}}
\end{fulllineitems}

\index{media (microdata.admin.CircuitTypeAdmin attribute)}

\begin{fulllineitems}
\phantomsection\label{modules/microdata:microdata.admin.CircuitTypeAdmin.media}\pysigline{\bfcode{media}}
\end{fulllineitems}


\end{fulllineitems}

\index{DeviceAdmin (class in microdata.admin)}

\begin{fulllineitems}
\phantomsection\label{modules/microdata:microdata.admin.DeviceAdmin}\pysiglinewithargsret{\strong{class }\code{microdata.admin.}\bfcode{DeviceAdmin}}{\emph{model}, \emph{admin\_site}}{}
Bases: \code{django.contrib.admin.options.ModelAdmin}
\index{inlines (microdata.admin.DeviceAdmin attribute)}

\begin{fulllineitems}
\phantomsection\label{modules/microdata:microdata.admin.DeviceAdmin.inlines}\pysigline{\bfcode{inlines}\strong{ = (\textless{}class `microdata.admin.DeviceWebSettingsInline'\textgreater{}, \textless{}class `farmer.admin.DeviceSettingsInline'\textgreater{})}}
\end{fulllineitems}

\index{list\_display (microdata.admin.DeviceAdmin attribute)}

\begin{fulllineitems}
\phantomsection\label{modules/microdata:microdata.admin.DeviceAdmin.list_display}\pysigline{\bfcode{list\_display}\strong{ = (`name', `owner', `serial', `position', `secret\_key', `registered', `fanout\_query\_registered')}}
\end{fulllineitems}

\index{media (microdata.admin.DeviceAdmin attribute)}

\begin{fulllineitems}
\phantomsection\label{modules/microdata:microdata.admin.DeviceAdmin.media}\pysigline{\bfcode{media}}
\end{fulllineitems}

\index{readonly\_fields (microdata.admin.DeviceAdmin attribute)}

\begin{fulllineitems}
\phantomsection\label{modules/microdata:microdata.admin.DeviceAdmin.readonly_fields}\pysigline{\bfcode{readonly\_fields}\strong{ = (`secret\_key',)}}
\end{fulllineitems}

\index{search\_fields (microdata.admin.DeviceAdmin attribute)}

\begin{fulllineitems}
\phantomsection\label{modules/microdata:microdata.admin.DeviceAdmin.search_fields}\pysigline{\bfcode{search\_fields}\strong{ = (`name', `serial')}}
\end{fulllineitems}


\end{fulllineitems}

\index{DeviceWebSettingsInline (class in microdata.admin)}

\begin{fulllineitems}
\phantomsection\label{modules/microdata:microdata.admin.DeviceWebSettingsInline}\pysiglinewithargsret{\strong{class }\code{microdata.admin.}\bfcode{DeviceWebSettingsInline}}{\emph{parent\_model}, \emph{admin\_site}}{}
Bases: \code{django.contrib.admin.options.StackedInline}
\index{can\_delete (microdata.admin.DeviceWebSettingsInline attribute)}

\begin{fulllineitems}
\phantomsection\label{modules/microdata:microdata.admin.DeviceWebSettingsInline.can_delete}\pysigline{\bfcode{can\_delete}\strong{ = False}}
\end{fulllineitems}

\index{media (microdata.admin.DeviceWebSettingsInline attribute)}

\begin{fulllineitems}
\phantomsection\label{modules/microdata:microdata.admin.DeviceWebSettingsInline.media}\pysigline{\bfcode{media}}
\end{fulllineitems}

\index{model (microdata.admin.DeviceWebSettingsInline attribute)}

\begin{fulllineitems}
\phantomsection\label{modules/microdata:microdata.admin.DeviceWebSettingsInline.model}\pysigline{\bfcode{model}}
alias of \code{DeviceWebSettings}

\end{fulllineitems}

\index{verbose\_name\_plural (microdata.admin.DeviceWebSettingsInline attribute)}

\begin{fulllineitems}
\phantomsection\label{modules/microdata:microdata.admin.DeviceWebSettingsInline.verbose_name_plural}\pysigline{\bfcode{verbose\_name\_plural}\strong{ = `devicesettings'}}
\end{fulllineitems}


\end{fulllineitems}



\subsection{microdata.models module}
\label{modules/microdata:microdata-models-module}\phantomsection\label{modules/microdata:module-microdata.models}\index{microdata.models (module)}\index{Appliance (class in microdata.models)}

\begin{fulllineitems}
\phantomsection\label{modules/microdata:microdata.models.Appliance}\pysiglinewithargsret{\strong{class }\code{microdata.models.}\bfcode{Appliance}}{\emph{*args}, \emph{**kwargs}}{}
Bases: \code{django.db.models.base.Model}
\index{Appliance.DoesNotExist}

\begin{fulllineitems}
\phantomsection\label{modules/microdata:microdata.models.Appliance.DoesNotExist}\pysigline{\strong{exception }\bfcode{DoesNotExist}}
Bases: \code{django.core.exceptions.ObjectDoesNotExist}

\end{fulllineitems}

\index{Appliance.MultipleObjectsReturned}

\begin{fulllineitems}
\phantomsection\label{modules/microdata:microdata.models.Appliance.MultipleObjectsReturned}\pysigline{\strong{exception }\code{Appliance.}\bfcode{MultipleObjectsReturned}}
Bases: \code{django.core.exceptions.MultipleObjectsReturned}

\end{fulllineitems}

\index{circuittype\_set (microdata.models.Appliance attribute)}

\begin{fulllineitems}
\phantomsection\label{modules/microdata:microdata.models.Appliance.circuittype_set}\pysigline{\code{Appliance.}\bfcode{circuittype\_set}}
\end{fulllineitems}

\index{eventnotification\_set (microdata.models.Appliance attribute)}

\begin{fulllineitems}
\phantomsection\label{modules/microdata:microdata.models.Appliance.eventnotification_set}\pysigline{\code{Appliance.}\bfcode{eventnotification\_set}}
\end{fulllineitems}

\index{objects (microdata.models.Appliance attribute)}

\begin{fulllineitems}
\phantomsection\label{modules/microdata:microdata.models.Appliance.objects}\pysigline{\code{Appliance.}\bfcode{objects}\strong{ = \textless{}django.db.models.manager.Manager object\textgreater{}}}
\end{fulllineitems}


\end{fulllineitems}

\index{Circuit (class in microdata.models)}

\begin{fulllineitems}
\phantomsection\label{modules/microdata:microdata.models.Circuit}\pysiglinewithargsret{\strong{class }\code{microdata.models.}\bfcode{Circuit}}{\emph{id}, \emph{circuittype\_id}, \emph{name}}{}
Bases: \code{django.db.models.base.Model}
\index{Circuit.DoesNotExist}

\begin{fulllineitems}
\phantomsection\label{modules/microdata:microdata.models.Circuit.DoesNotExist}\pysigline{\strong{exception }\bfcode{DoesNotExist}}
Bases: \code{django.core.exceptions.ObjectDoesNotExist}

\end{fulllineitems}

\index{Circuit.MultipleObjectsReturned}

\begin{fulllineitems}
\phantomsection\label{modules/microdata:microdata.models.Circuit.MultipleObjectsReturned}\pysigline{\strong{exception }\code{Circuit.}\bfcode{MultipleObjectsReturned}}
Bases: \code{django.core.exceptions.MultipleObjectsReturned}

\end{fulllineitems}

\index{circuittype (microdata.models.Circuit attribute)}

\begin{fulllineitems}
\phantomsection\label{modules/microdata:microdata.models.Circuit.circuittype}\pysigline{\code{Circuit.}\bfcode{circuittype}}
\end{fulllineitems}

\index{objects (microdata.models.Circuit attribute)}

\begin{fulllineitems}
\phantomsection\label{modules/microdata:microdata.models.Circuit.objects}\pysigline{\code{Circuit.}\bfcode{objects}\strong{ = \textless{}django.db.models.manager.Manager object\textgreater{}}}
\end{fulllineitems}


\end{fulllineitems}

\index{CircuitType (class in microdata.models)}

\begin{fulllineitems}
\phantomsection\label{modules/microdata:microdata.models.CircuitType}\pysiglinewithargsret{\strong{class }\code{microdata.models.}\bfcode{CircuitType}}{\emph{id}, \emph{name}, \emph{chart\_color}}{}
Bases: \code{django.db.models.base.Model}
\index{CircuitType.DoesNotExist}

\begin{fulllineitems}
\phantomsection\label{modules/microdata:microdata.models.CircuitType.DoesNotExist}\pysigline{\strong{exception }\bfcode{DoesNotExist}}
Bases: \code{django.core.exceptions.ObjectDoesNotExist}

\end{fulllineitems}

\index{CircuitType.MultipleObjectsReturned}

\begin{fulllineitems}
\phantomsection\label{modules/microdata:microdata.models.CircuitType.MultipleObjectsReturned}\pysigline{\strong{exception }\code{CircuitType.}\bfcode{MultipleObjectsReturned}}
Bases: \code{django.core.exceptions.MultipleObjectsReturned}

\end{fulllineitems}

\index{appliances (microdata.models.CircuitType attribute)}

\begin{fulllineitems}
\phantomsection\label{modules/microdata:microdata.models.CircuitType.appliances}\pysigline{\code{CircuitType.}\bfcode{appliances}}
\end{fulllineitems}

\index{circuit\_set (microdata.models.CircuitType attribute)}

\begin{fulllineitems}
\phantomsection\label{modules/microdata:microdata.models.CircuitType.circuit_set}\pysigline{\code{CircuitType.}\bfcode{circuit\_set}}
\end{fulllineitems}

\index{objects (microdata.models.CircuitType attribute)}

\begin{fulllineitems}
\phantomsection\label{modules/microdata:microdata.models.CircuitType.objects}\pysigline{\code{CircuitType.}\bfcode{objects}\strong{ = \textless{}django.db.models.manager.Manager object\textgreater{}}}
\end{fulllineitems}


\end{fulllineitems}

\index{Device (class in microdata.models)}

\begin{fulllineitems}
\phantomsection\label{modules/microdata:microdata.models.Device}\pysiglinewithargsret{\strong{class }\code{microdata.models.}\bfcode{Device}}{\emph{owner\_id}, \emph{ip\_address}, \emph{secret\_key}, \emph{serial}, \emph{name}, \emph{position}, \emph{registered}, \emph{fanout\_query\_registered}, \emph{channel\_1\_id}, \emph{channel\_2\_id}, \emph{channel\_3\_id}, \emph{data\_retention\_policy}, \emph{kilowatt\_hours\_monthly}, \emph{kilowatt\_hours\_daily}, \emph{cost\_daily}}{}
Bases: \code{django.db.models.base.Model}
\index{Device.DoesNotExist}

\begin{fulllineitems}
\phantomsection\label{modules/microdata:microdata.models.Device.DoesNotExist}\pysigline{\strong{exception }\bfcode{DoesNotExist}}
Bases: \code{django.core.exceptions.ObjectDoesNotExist}

\end{fulllineitems}

\index{Device.MultipleObjectsReturned}

\begin{fulllineitems}
\phantomsection\label{modules/microdata:microdata.models.Device.MultipleObjectsReturned}\pysigline{\strong{exception }\code{Device.}\bfcode{MultipleObjectsReturned}}
Bases: \code{django.core.exceptions.MultipleObjectsReturned}

\end{fulllineitems}

\index{channel\_1 (microdata.models.Device attribute)}

\begin{fulllineitems}
\phantomsection\label{modules/microdata:microdata.models.Device.channel_1}\pysigline{\code{Device.}\bfcode{channel\_1}}
\end{fulllineitems}

\index{channel\_2 (microdata.models.Device attribute)}

\begin{fulllineitems}
\phantomsection\label{modules/microdata:microdata.models.Device.channel_2}\pysigline{\code{Device.}\bfcode{channel\_2}}
\end{fulllineitems}

\index{channel\_3 (microdata.models.Device attribute)}

\begin{fulllineitems}
\phantomsection\label{modules/microdata:microdata.models.Device.channel_3}\pysigline{\code{Device.}\bfcode{channel\_3}}
\end{fulllineitems}

\index{delete() (microdata.models.Device method)}

\begin{fulllineitems}
\phantomsection\label{modules/microdata:microdata.models.Device.delete}\pysiglinewithargsret{\code{Device.}\bfcode{delete}}{\emph{*args}, \emph{**kwargs}}{}
\end{fulllineitems}

\index{devicesettings (microdata.models.Device attribute)}

\begin{fulllineitems}
\phantomsection\label{modules/microdata:microdata.models.Device.devicesettings}\pysigline{\code{Device.}\bfcode{devicesettings}}
\end{fulllineitems}

\index{devicewebsettings (microdata.models.Device attribute)}

\begin{fulllineitems}
\phantomsection\label{modules/microdata:microdata.models.Device.devicewebsettings}\pysigline{\code{Device.}\bfcode{devicewebsettings}}
\end{fulllineitems}

\index{event\_set (microdata.models.Device attribute)}

\begin{fulllineitems}
\phantomsection\label{modules/microdata:microdata.models.Device.event_set}\pysigline{\code{Device.}\bfcode{event\_set}}
\end{fulllineitems}

\index{objects (microdata.models.Device attribute)}

\begin{fulllineitems}
\phantomsection\label{modules/microdata:microdata.models.Device.objects}\pysigline{\code{Device.}\bfcode{objects}\strong{ = \textless{}django.db.models.manager.Manager object\textgreater{}}}
\end{fulllineitems}

\index{owner (microdata.models.Device attribute)}

\begin{fulllineitems}
\phantomsection\label{modules/microdata:microdata.models.Device.owner}\pysigline{\code{Device.}\bfcode{owner}}
\end{fulllineitems}

\index{save() (microdata.models.Device method)}

\begin{fulllineitems}
\phantomsection\label{modules/microdata:microdata.models.Device.save}\pysiglinewithargsret{\code{Device.}\bfcode{save}}{\emph{**kwargs}}{}
\end{fulllineitems}

\index{share\_with (microdata.models.Device attribute)}

\begin{fulllineitems}
\phantomsection\label{modules/microdata:microdata.models.Device.share_with}\pysigline{\code{Device.}\bfcode{share\_with}}
\end{fulllineitems}


\end{fulllineitems}

\index{Event (class in microdata.models)}

\begin{fulllineitems}
\phantomsection\label{modules/microdata:microdata.models.Event}\pysiglinewithargsret{\strong{class }\code{microdata.models.}\bfcode{Event}}{\emph{id}, \emph{device\_id}, \emph{dataPoints}, \emph{start}, \emph{frequency}, \emph{query}}{}
Bases: \code{django.db.models.base.Model}
\index{Event.DoesNotExist}

\begin{fulllineitems}
\phantomsection\label{modules/microdata:microdata.models.Event.DoesNotExist}\pysigline{\strong{exception }\bfcode{DoesNotExist}}
Bases: \code{django.core.exceptions.ObjectDoesNotExist}

\end{fulllineitems}

\index{Event.MultipleObjectsReturned}

\begin{fulllineitems}
\phantomsection\label{modules/microdata:microdata.models.Event.MultipleObjectsReturned}\pysigline{\strong{exception }\code{Event.}\bfcode{MultipleObjectsReturned}}
Bases: \code{django.core.exceptions.MultipleObjectsReturned}

\end{fulllineitems}

\index{device (microdata.models.Event attribute)}

\begin{fulllineitems}
\phantomsection\label{modules/microdata:microdata.models.Event.device}\pysigline{\code{Event.}\bfcode{device}}
\end{fulllineitems}

\index{objects (microdata.models.Event attribute)}

\begin{fulllineitems}
\phantomsection\label{modules/microdata:microdata.models.Event.objects}\pysigline{\code{Event.}\bfcode{objects}\strong{ = \textless{}django.db.models.manager.Manager object\textgreater{}}}
\end{fulllineitems}

\index{save() (microdata.models.Event method)}

\begin{fulllineitems}
\phantomsection\label{modules/microdata:microdata.models.Event.save}\pysiglinewithargsret{\code{Event.}\bfcode{save}}{\emph{**kwargs}}{}
\end{fulllineitems}


\end{fulllineitems}



\subsection{microdata.serializers module}
\label{modules/microdata:microdata-serializers-module}\label{modules/microdata:module-microdata.serializers}\index{microdata.serializers (module)}\index{ApplianceSerializer (class in microdata.serializers)}

\begin{fulllineitems}
\phantomsection\label{modules/microdata:microdata.serializers.ApplianceSerializer}\pysiglinewithargsret{\strong{class }\code{microdata.serializers.}\bfcode{ApplianceSerializer}}{\emph{instance=None}, \emph{data=\textless{}class rest\_framework.fields.empty\textgreater{}}, \emph{**kwargs}}{}
Bases: \code{rest\_framework.serializers.HyperlinkedModelSerializer}
\index{ApplianceSerializer.Meta (class in microdata.serializers)}

\begin{fulllineitems}
\phantomsection\label{modules/microdata:microdata.serializers.ApplianceSerializer.Meta}\pysigline{\strong{class }\bfcode{Meta}}~\index{fields (microdata.serializers.ApplianceSerializer.Meta attribute)}

\begin{fulllineitems}
\phantomsection\label{modules/microdata:microdata.serializers.ApplianceSerializer.Meta.fields}\pysigline{\bfcode{fields}\strong{ = (`name', `serial', `chart\_color')}}
\end{fulllineitems}

\index{model (microdata.serializers.ApplianceSerializer.Meta attribute)}

\begin{fulllineitems}
\phantomsection\label{modules/microdata:microdata.serializers.ApplianceSerializer.Meta.model}\pysigline{\bfcode{model}}
alias of \code{Appliance}

\end{fulllineitems}


\end{fulllineitems}


\end{fulllineitems}

\index{CircuitSerializer (class in microdata.serializers)}

\begin{fulllineitems}
\phantomsection\label{modules/microdata:microdata.serializers.CircuitSerializer}\pysiglinewithargsret{\strong{class }\code{microdata.serializers.}\bfcode{CircuitSerializer}}{\emph{instance=None}, \emph{data=\textless{}class rest\_framework.fields.empty\textgreater{}}, \emph{**kwargs}}{}
Bases: \code{rest\_framework.serializers.HyperlinkedModelSerializer}
\index{CircuitSerializer.Meta (class in microdata.serializers)}

\begin{fulllineitems}
\phantomsection\label{modules/microdata:microdata.serializers.CircuitSerializer.Meta}\pysigline{\strong{class }\bfcode{Meta}}~\index{fields (microdata.serializers.CircuitSerializer.Meta attribute)}

\begin{fulllineitems}
\phantomsection\label{modules/microdata:microdata.serializers.CircuitSerializer.Meta.fields}\pysigline{\bfcode{fields}\strong{ = (`name', `appliances', `chart\_color', `pk')}}
\end{fulllineitems}

\index{model (microdata.serializers.CircuitSerializer.Meta attribute)}

\begin{fulllineitems}
\phantomsection\label{modules/microdata:microdata.serializers.CircuitSerializer.Meta.model}\pysigline{\bfcode{model}}
alias of \code{CircuitType}

\end{fulllineitems}


\end{fulllineitems}


\end{fulllineitems}

\index{DeviceSerializer (class in microdata.serializers)}

\begin{fulllineitems}
\phantomsection\label{modules/microdata:microdata.serializers.DeviceSerializer}\pysiglinewithargsret{\strong{class }\code{microdata.serializers.}\bfcode{DeviceSerializer}}{\emph{instance=None}, \emph{data=\textless{}class rest\_framework.fields.empty\textgreater{}}, \emph{**kwargs}}{}
Bases: \code{rest\_framework.serializers.HyperlinkedModelSerializer}
\index{DeviceSerializer.Meta (class in microdata.serializers)}

\begin{fulllineitems}
\phantomsection\label{modules/microdata:microdata.serializers.DeviceSerializer.Meta}\pysigline{\strong{class }\bfcode{Meta}}~\index{fields (microdata.serializers.DeviceSerializer.Meta attribute)}

\begin{fulllineitems}
\phantomsection\label{modules/microdata:microdata.serializers.DeviceSerializer.Meta.fields}\pysigline{\bfcode{fields}\strong{ = (`owner', `ip\_address', `secret\_key', `serial', `name', `registered', `fanout\_query\_registered', `channel\_1', `channel\_2', `channel\_3', `data\_retention\_policy', `kilowatt\_hours\_monthly', `kilowatt\_hours\_daily')}}
\end{fulllineitems}

\index{model (microdata.serializers.DeviceSerializer.Meta attribute)}

\begin{fulllineitems}
\phantomsection\label{modules/microdata:microdata.serializers.DeviceSerializer.Meta.model}\pysigline{\bfcode{model}}
alias of \code{Device}

\end{fulllineitems}


\end{fulllineitems}


\end{fulllineitems}

\index{EventSerializer (class in microdata.serializers)}

\begin{fulllineitems}
\phantomsection\label{modules/microdata:microdata.serializers.EventSerializer}\pysiglinewithargsret{\strong{class }\code{microdata.serializers.}\bfcode{EventSerializer}}{\emph{instance=None}, \emph{data=\textless{}class rest\_framework.fields.empty\textgreater{}}, \emph{**kwargs}}{}
Bases: \code{rest\_framework.serializers.HyperlinkedModelSerializer}
\index{EventSerializer.Meta (class in microdata.serializers)}

\begin{fulllineitems}
\phantomsection\label{modules/microdata:microdata.serializers.EventSerializer.Meta}\pysigline{\strong{class }\bfcode{Meta}}~\index{fields (microdata.serializers.EventSerializer.Meta attribute)}

\begin{fulllineitems}
\phantomsection\label{modules/microdata:microdata.serializers.EventSerializer.Meta.fields}\pysigline{\bfcode{fields}\strong{ = (`device', `dataPoints')}}
\end{fulllineitems}

\index{model (microdata.serializers.EventSerializer.Meta attribute)}

\begin{fulllineitems}
\phantomsection\label{modules/microdata:microdata.serializers.EventSerializer.Meta.model}\pysigline{\bfcode{model}}
alias of \code{Event}

\end{fulllineitems}


\end{fulllineitems}


\end{fulllineitems}



\subsection{microdata.tests module}
\label{modules/microdata:microdata-tests-module}\label{modules/microdata:module-microdata.tests}\index{microdata.tests (module)}

\subsection{microdata.views module}
\label{modules/microdata:microdata-views-module}\label{modules/microdata:module-microdata.views}\index{microdata.views (module)}\index{ApplianceViewSet (class in microdata.views)}

\begin{fulllineitems}
\phantomsection\label{modules/microdata:microdata.views.ApplianceViewSet}\pysiglinewithargsret{\strong{class }\code{microdata.views.}\bfcode{ApplianceViewSet}}{\emph{**kwargs}}{}
Bases: \code{rest\_framework.viewsets.ModelViewSet}

API endpoint that allows appliances to be viewed or edited.
\index{queryset (microdata.views.ApplianceViewSet attribute)}

\begin{fulllineitems}
\phantomsection\label{modules/microdata:microdata.views.ApplianceViewSet.queryset}\pysigline{\bfcode{queryset}}
\end{fulllineitems}

\index{serializer\_class (microdata.views.ApplianceViewSet attribute)}

\begin{fulllineitems}
\phantomsection\label{modules/microdata:microdata.views.ApplianceViewSet.serializer_class}\pysigline{\bfcode{serializer\_class}}
alias of \code{ApplianceSerializer}

\end{fulllineitems}


\end{fulllineitems}

\index{CircuitViewSet (class in microdata.views)}

\begin{fulllineitems}
\phantomsection\label{modules/microdata:microdata.views.CircuitViewSet}\pysiglinewithargsret{\strong{class }\code{microdata.views.}\bfcode{CircuitViewSet}}{\emph{**kwargs}}{}
Bases: \code{rest\_framework.viewsets.ModelViewSet}

API endpoint that allows circuits to be viewed or edited.
\index{queryset (microdata.views.CircuitViewSet attribute)}

\begin{fulllineitems}
\phantomsection\label{modules/microdata:microdata.views.CircuitViewSet.queryset}\pysigline{\bfcode{queryset}}
\end{fulllineitems}

\index{serializer\_class (microdata.views.CircuitViewSet attribute)}

\begin{fulllineitems}
\phantomsection\label{modules/microdata:microdata.views.CircuitViewSet.serializer_class}\pysigline{\bfcode{serializer\_class}}
alias of \code{CircuitSerializer}

\end{fulllineitems}


\end{fulllineitems}

\index{DeviceViewSet (class in microdata.views)}

\begin{fulllineitems}
\phantomsection\label{modules/microdata:microdata.views.DeviceViewSet}\pysiglinewithargsret{\strong{class }\code{microdata.views.}\bfcode{DeviceViewSet}}{\emph{**kwargs}}{}
Bases: \code{rest\_framework.viewsets.ModelViewSet}

API endpoint that allows devices to be viewed or edited.
\index{create() (microdata.views.DeviceViewSet method)}

\begin{fulllineitems}
\phantomsection\label{modules/microdata:microdata.views.DeviceViewSet.create}\pysiglinewithargsret{\bfcode{create}}{\emph{request}}{}
\end{fulllineitems}

\index{queryset (microdata.views.DeviceViewSet attribute)}

\begin{fulllineitems}
\phantomsection\label{modules/microdata:microdata.views.DeviceViewSet.queryset}\pysigline{\bfcode{queryset}}
\end{fulllineitems}

\index{serializer\_class (microdata.views.DeviceViewSet attribute)}

\begin{fulllineitems}
\phantomsection\label{modules/microdata:microdata.views.DeviceViewSet.serializer_class}\pysigline{\bfcode{serializer\_class}}
alias of \code{DeviceSerializer}

\end{fulllineitems}


\end{fulllineitems}

\index{EventViewSet (class in microdata.views)}

\begin{fulllineitems}
\phantomsection\label{modules/microdata:microdata.views.EventViewSet}\pysiglinewithargsret{\strong{class }\code{microdata.views.}\bfcode{EventViewSet}}{\emph{**kwargs}}{}
Bases: \code{rest\_framework.viewsets.ModelViewSet}

API endpoint that allows events to be viewed or edited.
\index{create() (microdata.views.EventViewSet method)}

\begin{fulllineitems}
\phantomsection\label{modules/microdata:microdata.views.EventViewSet.create}\pysiglinewithargsret{\bfcode{create}}{\emph{request}}{}
\end{fulllineitems}

\index{queryset (microdata.views.EventViewSet attribute)}

\begin{fulllineitems}
\phantomsection\label{modules/microdata:microdata.views.EventViewSet.queryset}\pysigline{\bfcode{queryset}}
\end{fulllineitems}

\index{serializer\_class (microdata.views.EventViewSet attribute)}

\begin{fulllineitems}
\phantomsection\label{modules/microdata:microdata.views.EventViewSet.serializer_class}\pysigline{\bfcode{serializer\_class}}
alias of \code{EventSerializer}

\end{fulllineitems}


\end{fulllineitems}

\index{KeyForm (class in microdata.views)}

\begin{fulllineitems}
\phantomsection\label{modules/microdata:microdata.views.KeyForm}\pysiglinewithargsret{\strong{class }\code{microdata.views.}\bfcode{KeyForm}}{\emph{data=None}, \emph{files=None}, \emph{auto\_id=u'id\_\%s'}, \emph{prefix=None}, \emph{initial=None}, \emph{error\_class=\textless{}class `django.forms.util.ErrorList'\textgreater{}}, \emph{label\_suffix=None}, \emph{empty\_permitted=False}}{}
Bases: \code{django.forms.forms.Form}

Form class used to generate a simple key form. Currently used when a user intends to add a new device via the webapp interface.
\index{base\_fields (microdata.views.KeyForm attribute)}

\begin{fulllineitems}
\phantomsection\label{modules/microdata:microdata.views.KeyForm.base_fields}\pysigline{\bfcode{base\_fields}\strong{ = \{`serial': \textless{}django.forms.fields.IntegerField object at 0x2b6d59e1a390\textgreater{}\}}}
\end{fulllineitems}

\index{media (microdata.views.KeyForm attribute)}

\begin{fulllineitems}
\phantomsection\label{modules/microdata:microdata.views.KeyForm.media}\pysigline{\bfcode{media}}
\end{fulllineitems}


\end{fulllineitems}

\index{initiate\_job\_to\_glacier() (in module microdata.views)}

\begin{fulllineitems}
\phantomsection\label{modules/microdata:microdata.views.initiate_job_to_glacier}\pysiglinewithargsret{\code{microdata.views.}\bfcode{initiate\_job\_to\_glacier}}{\emph{request}, \emph{requester}, \emph{end\_time}}{}
Experimental class to demonstrate the possibility to archive old data to Amazon Glacier.

Requires an Amazon AWS key to be set in the environment variables.

\end{fulllineitems}

\index{new\_device() (in module microdata.views)}

\begin{fulllineitems}
\phantomsection\label{modules/microdata:microdata.views.new_device}\pysiglinewithargsret{\code{microdata.views.}\bfcode{new\_device}}{\emph{request}}{}
Function used to service a user's request to add a new device.
\begin{description}
\item[{Args:}] \leavevmode
request (HttpRequest): Request object from the user

\item[{Returns:}] \leavevmode
render with context:

\begin{Verbatim}[commandchars=\\\{\}]
\PYGZsq{}form\PYGZsq{}   : form object if user requests the form
\PYGZsq{}error\PYGZsq{}  : string \PYGZhy{} error description if present
\PYGZsq{}created\PYGZsq{}: true/false if device is created
\PYGZsq{}device\PYGZsq{} : serialized device object if device is created
\end{Verbatim}

\end{description}

\end{fulllineitems}

\index{timestamp() (in module microdata.views)}

\begin{fulllineitems}
\phantomsection\label{modules/microdata:microdata.views.timestamp}\pysiglinewithargsret{\code{microdata.views.}\bfcode{timestamp}}{\emph{request}}{}
Function to return the server's time in milliseconds.

This function is possibly deprecated. Devices should now get the server time
from \code{farmer.DeviceSettingsViewSet}.

\end{fulllineitems}



\subsection{Module contents}
\label{modules/microdata:module-contents}\label{modules/microdata:module-microdata}\index{microdata (module)}

\section{webapp package}
\label{modules/webapp::doc}\label{modules/webapp:webapp-package}

\subsection{Subpackages}
\label{modules/webapp:subpackages}

\subsubsection{webapp.management package}
\label{modules/webapp.management::doc}\label{modules/webapp.management:webapp-management-package}

\paragraph{Subpackages}
\label{modules/webapp.management:subpackages}

\subparagraph{webapp.management.commands package}
\label{modules/webapp.management.commands::doc}\label{modules/webapp.management.commands:webapp-management-commands-package}

\subparagraph{Submodules}
\label{modules/webapp.management.commands:submodules}

\subparagraph{webapp.management.commands.email\_event module}
\label{modules/webapp.management.commands:webapp-management-commands-email-event-module}

\subparagraph{webapp.management.commands.email\_interval module}
\label{modules/webapp.management.commands:webapp-management-commands-email-interval-module}

\subparagraph{webapp.management.commands.reset\_kilowatt\_accumulations module}
\label{modules/webapp.management.commands:module-webapp.management.commands.reset_kilowatt_accumulations}\label{modules/webapp.management.commands:webapp-management-commands-reset-kilowatt-accumulations-module}\index{webapp.management.commands.reset\_kilowatt\_accumulations (module)}\index{Command (class in webapp.management.commands.reset\_kilowatt\_accumulations)}

\begin{fulllineitems}
\phantomsection\label{modules/webapp.management.commands:webapp.management.commands.reset_kilowatt_accumulations.Command}\pysigline{\strong{class }\code{webapp.management.commands.reset\_kilowatt\_accumulations.}\bfcode{Command}}
Bases: \code{django.core.management.base.BaseCommand}
\index{args (webapp.management.commands.reset\_kilowatt\_accumulations.Command attribute)}

\begin{fulllineitems}
\phantomsection\label{modules/webapp.management.commands:webapp.management.commands.reset_kilowatt_accumulations.Command.args}\pysigline{\bfcode{args}\strong{ = `daily, weekly'}}
\end{fulllineitems}

\index{handle() (webapp.management.commands.reset\_kilowatt\_accumulations.Command method)}

\begin{fulllineitems}
\phantomsection\label{modules/webapp.management.commands:webapp.management.commands.reset_kilowatt_accumulations.Command.handle}\pysiglinewithargsret{\bfcode{handle}}{\emph{*args}, \emph{**options}}{}
\end{fulllineitems}

\index{help (webapp.management.commands.reset\_kilowatt\_accumulations.Command attribute)}

\begin{fulllineitems}
\phantomsection\label{modules/webapp.management.commands:webapp.management.commands.reset_kilowatt_accumulations.Command.help}\pysigline{\bfcode{help}\strong{ = `'}}
\end{fulllineitems}


\end{fulllineitems}



\subparagraph{Module contents}
\label{modules/webapp.management.commands:module-webapp.management.commands}\label{modules/webapp.management.commands:module-contents}\index{webapp.management.commands (module)}

\paragraph{Module contents}
\label{modules/webapp.management:module-contents}\label{modules/webapp.management:module-webapp.management}\index{webapp.management (module)}

\subsection{Submodules}
\label{modules/webapp:submodules}

\subsection{webapp.admin module}
\label{modules/webapp:module-webapp.admin}\label{modules/webapp:webapp-admin-module}\index{webapp.admin (module)}\index{DashboardSettingsInline (class in webapp.admin)}

\begin{fulllineitems}
\phantomsection\label{modules/webapp:webapp.admin.DashboardSettingsInline}\pysiglinewithargsret{\strong{class }\code{webapp.admin.}\bfcode{DashboardSettingsInline}}{\emph{parent\_model}, \emph{admin\_site}}{}
Bases: \code{django.contrib.admin.options.StackedInline}
\index{can\_delete (webapp.admin.DashboardSettingsInline attribute)}

\begin{fulllineitems}
\phantomsection\label{modules/webapp:webapp.admin.DashboardSettingsInline.can_delete}\pysigline{\bfcode{can\_delete}\strong{ = False}}
\end{fulllineitems}

\index{media (webapp.admin.DashboardSettingsInline attribute)}

\begin{fulllineitems}
\phantomsection\label{modules/webapp:webapp.admin.DashboardSettingsInline.media}\pysigline{\bfcode{media}}
\end{fulllineitems}

\index{model (webapp.admin.DashboardSettingsInline attribute)}

\begin{fulllineitems}
\phantomsection\label{modules/webapp:webapp.admin.DashboardSettingsInline.model}\pysigline{\bfcode{model}}
alias of \code{DashboardSettings}

\end{fulllineitems}

\index{verbose\_name\_plural (webapp.admin.DashboardSettingsInline attribute)}

\begin{fulllineitems}
\phantomsection\label{modules/webapp:webapp.admin.DashboardSettingsInline.verbose_name_plural}\pysigline{\bfcode{verbose\_name\_plural}\strong{ = `dashboardsettings'}}
\end{fulllineitems}


\end{fulllineitems}

\index{RatePlanAdmin (class in webapp.admin)}

\begin{fulllineitems}
\phantomsection\label{modules/webapp:webapp.admin.RatePlanAdmin}\pysiglinewithargsret{\strong{class }\code{webapp.admin.}\bfcode{RatePlanAdmin}}{\emph{model}, \emph{admin\_site}}{}
Bases: \code{django.contrib.admin.options.ModelAdmin}
\index{inlines (webapp.admin.RatePlanAdmin attribute)}

\begin{fulllineitems}
\phantomsection\label{modules/webapp:webapp.admin.RatePlanAdmin.inlines}\pysigline{\bfcode{inlines}\strong{ = (\textless{}class `webapp.admin.TierInline'\textgreater{},)}}
\end{fulllineitems}

\index{list\_display (webapp.admin.RatePlanAdmin attribute)}

\begin{fulllineitems}
\phantomsection\label{modules/webapp:webapp.admin.RatePlanAdmin.list_display}\pysigline{\bfcode{list\_display}\strong{ = (`description', `utility\_company', `pk')}}
\end{fulllineitems}

\index{media (webapp.admin.RatePlanAdmin attribute)}

\begin{fulllineitems}
\phantomsection\label{modules/webapp:webapp.admin.RatePlanAdmin.media}\pysigline{\bfcode{media}}
\end{fulllineitems}


\end{fulllineitems}

\index{TerritoryAdmin (class in webapp.admin)}

\begin{fulllineitems}
\phantomsection\label{modules/webapp:webapp.admin.TerritoryAdmin}\pysiglinewithargsret{\strong{class }\code{webapp.admin.}\bfcode{TerritoryAdmin}}{\emph{model}, \emph{admin\_site}}{}
Bases: \code{django.contrib.admin.options.ModelAdmin}
\index{list\_display (webapp.admin.TerritoryAdmin attribute)}

\begin{fulllineitems}
\phantomsection\label{modules/webapp:webapp.admin.TerritoryAdmin.list_display}\pysigline{\bfcode{list\_display}\strong{ = (`description', `rate\_plan', `pk')}}
\end{fulllineitems}

\index{media (webapp.admin.TerritoryAdmin attribute)}

\begin{fulllineitems}
\phantomsection\label{modules/webapp:webapp.admin.TerritoryAdmin.media}\pysigline{\bfcode{media}}
\end{fulllineitems}


\end{fulllineitems}

\index{TierInline (class in webapp.admin)}

\begin{fulllineitems}
\phantomsection\label{modules/webapp:webapp.admin.TierInline}\pysiglinewithargsret{\strong{class }\code{webapp.admin.}\bfcode{TierInline}}{\emph{parent\_model}, \emph{admin\_site}}{}
Bases: \code{django.contrib.admin.options.StackedInline}
\index{media (webapp.admin.TierInline attribute)}

\begin{fulllineitems}
\phantomsection\label{modules/webapp:webapp.admin.TierInline.media}\pysigline{\bfcode{media}}
\end{fulllineitems}

\index{model (webapp.admin.TierInline attribute)}

\begin{fulllineitems}
\phantomsection\label{modules/webapp:webapp.admin.TierInline.model}\pysigline{\bfcode{model}}
alias of \code{Tier}

\end{fulllineitems}


\end{fulllineitems}

\index{UserAdmin (class in webapp.admin)}

\begin{fulllineitems}
\phantomsection\label{modules/webapp:webapp.admin.UserAdmin}\pysiglinewithargsret{\strong{class }\code{webapp.admin.}\bfcode{UserAdmin}}{\emph{model}, \emph{admin\_site}}{}
Bases: \code{django.contrib.auth.admin.UserAdmin}
\index{inlines (webapp.admin.UserAdmin attribute)}

\begin{fulllineitems}
\phantomsection\label{modules/webapp:webapp.admin.UserAdmin.inlines}\pysigline{\bfcode{inlines}\strong{ = (\textless{}class `webapp.admin.UserSettingsInline'\textgreater{}, \textless{}class `webapp.admin.DashboardSettingsInline'\textgreater{})}}
\end{fulllineitems}

\index{media (webapp.admin.UserAdmin attribute)}

\begin{fulllineitems}
\phantomsection\label{modules/webapp:webapp.admin.UserAdmin.media}\pysigline{\bfcode{media}}
\end{fulllineitems}


\end{fulllineitems}

\index{UserSettingsInline (class in webapp.admin)}

\begin{fulllineitems}
\phantomsection\label{modules/webapp:webapp.admin.UserSettingsInline}\pysiglinewithargsret{\strong{class }\code{webapp.admin.}\bfcode{UserSettingsInline}}{\emph{parent\_model}, \emph{admin\_site}}{}
Bases: \code{django.contrib.admin.options.StackedInline}
\index{can\_delete (webapp.admin.UserSettingsInline attribute)}

\begin{fulllineitems}
\phantomsection\label{modules/webapp:webapp.admin.UserSettingsInline.can_delete}\pysigline{\bfcode{can\_delete}\strong{ = False}}
\end{fulllineitems}

\index{media (webapp.admin.UserSettingsInline attribute)}

\begin{fulllineitems}
\phantomsection\label{modules/webapp:webapp.admin.UserSettingsInline.media}\pysigline{\bfcode{media}}
\end{fulllineitems}

\index{model (webapp.admin.UserSettingsInline attribute)}

\begin{fulllineitems}
\phantomsection\label{modules/webapp:webapp.admin.UserSettingsInline.model}\pysigline{\bfcode{model}}
alias of \code{UserSettings}

\end{fulllineitems}

\index{verbose\_name\_plural (webapp.admin.UserSettingsInline attribute)}

\begin{fulllineitems}
\phantomsection\label{modules/webapp:webapp.admin.UserSettingsInline.verbose_name_plural}\pysigline{\bfcode{verbose\_name\_plural}\strong{ = `usersettings'}}
\end{fulllineitems}


\end{fulllineitems}

\index{UtilityCompanyAdmin (class in webapp.admin)}

\begin{fulllineitems}
\phantomsection\label{modules/webapp:webapp.admin.UtilityCompanyAdmin}\pysiglinewithargsret{\strong{class }\code{webapp.admin.}\bfcode{UtilityCompanyAdmin}}{\emph{model}, \emph{admin\_site}}{}
Bases: \code{django.contrib.admin.options.ModelAdmin}
\index{list\_display (webapp.admin.UtilityCompanyAdmin attribute)}

\begin{fulllineitems}
\phantomsection\label{modules/webapp:webapp.admin.UtilityCompanyAdmin.list_display}\pysigline{\bfcode{list\_display}\strong{ = (`description', `pk')}}
\end{fulllineitems}

\index{media (webapp.admin.UtilityCompanyAdmin attribute)}

\begin{fulllineitems}
\phantomsection\label{modules/webapp:webapp.admin.UtilityCompanyAdmin.media}\pysigline{\bfcode{media}}
\end{fulllineitems}


\end{fulllineitems}



\subsection{webapp.device\_dictionary module}
\label{modules/webapp:module-webapp.device_dictionary}\label{modules/webapp:webapp-device-dictionary-module}\index{webapp.device\_dictionary (module)}

\subsection{webapp.models module}
\label{modules/webapp:webapp-models-module}\phantomsection\label{modules/webapp:module-webapp.models}\index{webapp.models (module)}\index{DashboardSettings (class in webapp.models)}

\begin{fulllineitems}
\phantomsection\label{modules/webapp:webapp.models.DashboardSettings}\pysiglinewithargsret{\strong{class }\code{webapp.models.}\bfcode{DashboardSettings}}{\emph{id}, \emph{user\_id}, \emph{stack}}{}
Bases: \code{django.db.models.base.Model}
\index{DashboardSettings.DoesNotExist}

\begin{fulllineitems}
\phantomsection\label{modules/webapp:webapp.models.DashboardSettings.DoesNotExist}\pysigline{\strong{exception }\bfcode{DoesNotExist}}
Bases: \code{django.core.exceptions.ObjectDoesNotExist}

\end{fulllineitems}

\index{DashboardSettings.MultipleObjectsReturned}

\begin{fulllineitems}
\phantomsection\label{modules/webapp:webapp.models.DashboardSettings.MultipleObjectsReturned}\pysigline{\strong{exception }\code{DashboardSettings.}\bfcode{MultipleObjectsReturned}}
Bases: \code{django.core.exceptions.MultipleObjectsReturned}

\end{fulllineitems}

\index{objects (webapp.models.DashboardSettings attribute)}

\begin{fulllineitems}
\phantomsection\label{modules/webapp:webapp.models.DashboardSettings.objects}\pysigline{\code{DashboardSettings.}\bfcode{objects}\strong{ = \textless{}django.db.models.manager.Manager object\textgreater{}}}
\end{fulllineitems}

\index{user (webapp.models.DashboardSettings attribute)}

\begin{fulllineitems}
\phantomsection\label{modules/webapp:webapp.models.DashboardSettings.user}\pysigline{\code{DashboardSettings.}\bfcode{user}}
\end{fulllineitems}


\end{fulllineitems}

\index{DeviceWebSettings (class in webapp.models)}

\begin{fulllineitems}
\phantomsection\label{modules/webapp:webapp.models.DeviceWebSettings}\pysiglinewithargsret{\strong{class }\code{webapp.models.}\bfcode{DeviceWebSettings}}{\emph{id}, \emph{device\_id}, \emph{current\_tier\_id}}{}
Bases: \code{django.db.models.base.Model}
\index{DeviceWebSettings.DoesNotExist}

\begin{fulllineitems}
\phantomsection\label{modules/webapp:webapp.models.DeviceWebSettings.DoesNotExist}\pysigline{\strong{exception }\bfcode{DoesNotExist}}
Bases: \code{django.core.exceptions.ObjectDoesNotExist}

\end{fulllineitems}

\index{DeviceWebSettings.MultipleObjectsReturned}

\begin{fulllineitems}
\phantomsection\label{modules/webapp:webapp.models.DeviceWebSettings.MultipleObjectsReturned}\pysigline{\strong{exception }\code{DeviceWebSettings.}\bfcode{MultipleObjectsReturned}}
Bases: \code{django.core.exceptions.MultipleObjectsReturned}

\end{fulllineitems}

\index{current\_tier (webapp.models.DeviceWebSettings attribute)}

\begin{fulllineitems}
\phantomsection\label{modules/webapp:webapp.models.DeviceWebSettings.current_tier}\pysigline{\code{DeviceWebSettings.}\bfcode{current\_tier}}
\end{fulllineitems}

\index{device (webapp.models.DeviceWebSettings attribute)}

\begin{fulllineitems}
\phantomsection\label{modules/webapp:webapp.models.DeviceWebSettings.device}\pysigline{\code{DeviceWebSettings.}\bfcode{device}}
\end{fulllineitems}

\index{objects (webapp.models.DeviceWebSettings attribute)}

\begin{fulllineitems}
\phantomsection\label{modules/webapp:webapp.models.DeviceWebSettings.objects}\pysigline{\code{DeviceWebSettings.}\bfcode{objects}\strong{ = \textless{}django.db.models.manager.Manager object\textgreater{}}}
\end{fulllineitems}

\index{rate\_plans (webapp.models.DeviceWebSettings attribute)}

\begin{fulllineitems}
\phantomsection\label{modules/webapp:webapp.models.DeviceWebSettings.rate_plans}\pysigline{\code{DeviceWebSettings.}\bfcode{rate\_plans}}
\end{fulllineitems}

\index{territories (webapp.models.DeviceWebSettings attribute)}

\begin{fulllineitems}
\phantomsection\label{modules/webapp:webapp.models.DeviceWebSettings.territories}\pysigline{\code{DeviceWebSettings.}\bfcode{territories}}
\end{fulllineitems}

\index{utility\_companies (webapp.models.DeviceWebSettings attribute)}

\begin{fulllineitems}
\phantomsection\label{modules/webapp:webapp.models.DeviceWebSettings.utility_companies}\pysigline{\code{DeviceWebSettings.}\bfcode{utility\_companies}}
\end{fulllineitems}


\end{fulllineitems}

\index{EventNotification (class in webapp.models)}

\begin{fulllineitems}
\phantomsection\label{modules/webapp:webapp.models.EventNotification}\pysiglinewithargsret{\strong{class }\code{webapp.models.}\bfcode{EventNotification}}{\emph{id}, \emph{description}, \emph{keyword}, \emph{watts\_above\_average}, \emph{period\_of\_time}, \emph{email\_subject}, \emph{email\_body}}{}
Bases: \code{django.db.models.base.Model}
\index{EventNotification.DoesNotExist}

\begin{fulllineitems}
\phantomsection\label{modules/webapp:webapp.models.EventNotification.DoesNotExist}\pysigline{\strong{exception }\bfcode{DoesNotExist}}
Bases: \code{django.core.exceptions.ObjectDoesNotExist}

\end{fulllineitems}

\index{EventNotification.MultipleObjectsReturned}

\begin{fulllineitems}
\phantomsection\label{modules/webapp:webapp.models.EventNotification.MultipleObjectsReturned}\pysigline{\strong{exception }\code{EventNotification.}\bfcode{MultipleObjectsReturned}}
Bases: \code{django.core.exceptions.MultipleObjectsReturned}

\end{fulllineitems}

\index{appliances\_to\_watch (webapp.models.EventNotification attribute)}

\begin{fulllineitems}
\phantomsection\label{modules/webapp:webapp.models.EventNotification.appliances_to_watch}\pysigline{\code{EventNotification.}\bfcode{appliances\_to\_watch}}
\end{fulllineitems}

\index{objects (webapp.models.EventNotification attribute)}

\begin{fulllineitems}
\phantomsection\label{modules/webapp:webapp.models.EventNotification.objects}\pysigline{\code{EventNotification.}\bfcode{objects}\strong{ = \textless{}django.db.models.manager.Manager object\textgreater{}}}
\end{fulllineitems}

\index{usersettings\_set (webapp.models.EventNotification attribute)}

\begin{fulllineitems}
\phantomsection\label{modules/webapp:webapp.models.EventNotification.usersettings_set}\pysigline{\code{EventNotification.}\bfcode{usersettings\_set}}
\end{fulllineitems}


\end{fulllineitems}

\index{IntervalNotification (class in webapp.models)}

\begin{fulllineitems}
\phantomsection\label{modules/webapp:webapp.models.IntervalNotification}\pysiglinewithargsret{\strong{class }\code{webapp.models.}\bfcode{IntervalNotification}}{\emph{id}, \emph{description}, \emph{recurrences}, \emph{email\_subject}, \emph{email\_body}}{}
Bases: \code{django.db.models.base.Model}
\index{IntervalNotification.DoesNotExist}

\begin{fulllineitems}
\phantomsection\label{modules/webapp:webapp.models.IntervalNotification.DoesNotExist}\pysigline{\strong{exception }\bfcode{DoesNotExist}}
Bases: \code{django.core.exceptions.ObjectDoesNotExist}

\end{fulllineitems}

\index{IntervalNotification.MultipleObjectsReturned}

\begin{fulllineitems}
\phantomsection\label{modules/webapp:webapp.models.IntervalNotification.MultipleObjectsReturned}\pysigline{\strong{exception }\code{IntervalNotification.}\bfcode{MultipleObjectsReturned}}
Bases: \code{django.core.exceptions.MultipleObjectsReturned}

\end{fulllineitems}

\index{notification\_set (webapp.models.IntervalNotification attribute)}

\begin{fulllineitems}
\phantomsection\label{modules/webapp:webapp.models.IntervalNotification.notification_set}\pysigline{\code{IntervalNotification.}\bfcode{notification\_set}}
\end{fulllineitems}

\index{objects (webapp.models.IntervalNotification attribute)}

\begin{fulllineitems}
\phantomsection\label{modules/webapp:webapp.models.IntervalNotification.objects}\pysigline{\code{IntervalNotification.}\bfcode{objects}\strong{ = \textless{}django.db.models.manager.Manager object\textgreater{}}}
\end{fulllineitems}

\index{usersettings\_set (webapp.models.IntervalNotification attribute)}

\begin{fulllineitems}
\phantomsection\label{modules/webapp:webapp.models.IntervalNotification.usersettings_set}\pysigline{\code{IntervalNotification.}\bfcode{usersettings\_set}}
\end{fulllineitems}


\end{fulllineitems}

\index{Notification (class in webapp.models)}

\begin{fulllineitems}
\phantomsection\label{modules/webapp:webapp.models.Notification}\pysiglinewithargsret{\strong{class }\code{webapp.models.}\bfcode{Notification}}{\emph{*args}, \emph{**kwargs}}{}
Bases: \code{django.db.models.base.Model}

DEPRECATED
\index{Notification.DoesNotExist}

\begin{fulllineitems}
\phantomsection\label{modules/webapp:webapp.models.Notification.DoesNotExist}\pysigline{\strong{exception }\bfcode{DoesNotExist}}
Bases: \code{django.core.exceptions.ObjectDoesNotExist}

\end{fulllineitems}

\index{Notification.MultipleObjectsReturned}

\begin{fulllineitems}
\phantomsection\label{modules/webapp:webapp.models.Notification.MultipleObjectsReturned}\pysigline{\strong{exception }\code{Notification.}\bfcode{MultipleObjectsReturned}}
Bases: \code{django.core.exceptions.MultipleObjectsReturned}

\end{fulllineitems}

\index{interval\_notification (webapp.models.Notification attribute)}

\begin{fulllineitems}
\phantomsection\label{modules/webapp:webapp.models.Notification.interval_notification}\pysigline{\code{Notification.}\bfcode{interval\_notification}}
\end{fulllineitems}

\index{objects (webapp.models.Notification attribute)}

\begin{fulllineitems}
\phantomsection\label{modules/webapp:webapp.models.Notification.objects}\pysigline{\code{Notification.}\bfcode{objects}\strong{ = \textless{}django.db.models.manager.Manager object\textgreater{}}}
\end{fulllineitems}

\index{user (webapp.models.Notification attribute)}

\begin{fulllineitems}
\phantomsection\label{modules/webapp:webapp.models.Notification.user}\pysigline{\code{Notification.}\bfcode{user}}
\end{fulllineitems}


\end{fulllineitems}

\index{RatePlan (class in webapp.models)}

\begin{fulllineitems}
\phantomsection\label{modules/webapp:webapp.models.RatePlan}\pysiglinewithargsret{\strong{class }\code{webapp.models.}\bfcode{RatePlan}}{\emph{id}, \emph{utility\_company\_id}, \emph{description}, \emph{data\_source}, \emph{min\_charge\_rate}, \emph{california\_climate\_credit}}{}
Bases: \code{django.db.models.base.Model}
\index{RatePlan.DoesNotExist}

\begin{fulllineitems}
\phantomsection\label{modules/webapp:webapp.models.RatePlan.DoesNotExist}\pysigline{\strong{exception }\bfcode{DoesNotExist}}
Bases: \code{django.core.exceptions.ObjectDoesNotExist}

\end{fulllineitems}

\index{RatePlan.MultipleObjectsReturned}

\begin{fulllineitems}
\phantomsection\label{modules/webapp:webapp.models.RatePlan.MultipleObjectsReturned}\pysigline{\strong{exception }\code{RatePlan.}\bfcode{MultipleObjectsReturned}}
Bases: \code{django.core.exceptions.MultipleObjectsReturned}

\end{fulllineitems}

\index{devicewebsettings\_set (webapp.models.RatePlan attribute)}

\begin{fulllineitems}
\phantomsection\label{modules/webapp:webapp.models.RatePlan.devicewebsettings_set}\pysigline{\code{RatePlan.}\bfcode{devicewebsettings\_set}}
\end{fulllineitems}

\index{objects (webapp.models.RatePlan attribute)}

\begin{fulllineitems}
\phantomsection\label{modules/webapp:webapp.models.RatePlan.objects}\pysigline{\code{RatePlan.}\bfcode{objects}\strong{ = \textless{}django.db.models.manager.Manager object\textgreater{}}}
\end{fulllineitems}

\index{territory\_set (webapp.models.RatePlan attribute)}

\begin{fulllineitems}
\phantomsection\label{modules/webapp:webapp.models.RatePlan.territory_set}\pysigline{\code{RatePlan.}\bfcode{territory\_set}}
\end{fulllineitems}

\index{tier\_set (webapp.models.RatePlan attribute)}

\begin{fulllineitems}
\phantomsection\label{modules/webapp:webapp.models.RatePlan.tier_set}\pysigline{\code{RatePlan.}\bfcode{tier\_set}}
\end{fulllineitems}

\index{utility\_company (webapp.models.RatePlan attribute)}

\begin{fulllineitems}
\phantomsection\label{modules/webapp:webapp.models.RatePlan.utility_company}\pysigline{\code{RatePlan.}\bfcode{utility\_company}}
\end{fulllineitems}


\end{fulllineitems}

\index{Territory (class in webapp.models)}

\begin{fulllineitems}
\phantomsection\label{modules/webapp:webapp.models.Territory}\pysiglinewithargsret{\strong{class }\code{webapp.models.}\bfcode{Territory}}{\emph{id}, \emph{rate\_plan\_id}, \emph{description}, \emph{data\_source}, \emph{summer\_start}, \emph{winter\_start}, \emph{summer\_rate}, \emph{winter\_rate}}{}
Bases: \code{django.db.models.base.Model}
\index{Territory.DoesNotExist}

\begin{fulllineitems}
\phantomsection\label{modules/webapp:webapp.models.Territory.DoesNotExist}\pysigline{\strong{exception }\bfcode{DoesNotExist}}
Bases: \code{django.core.exceptions.ObjectDoesNotExist}

\end{fulllineitems}

\index{Territory.MultipleObjectsReturned}

\begin{fulllineitems}
\phantomsection\label{modules/webapp:webapp.models.Territory.MultipleObjectsReturned}\pysigline{\strong{exception }\code{Territory.}\bfcode{MultipleObjectsReturned}}
Bases: \code{django.core.exceptions.MultipleObjectsReturned}

\end{fulllineitems}

\index{devicewebsettings\_set (webapp.models.Territory attribute)}

\begin{fulllineitems}
\phantomsection\label{modules/webapp:webapp.models.Territory.devicewebsettings_set}\pysigline{\code{Territory.}\bfcode{devicewebsettings\_set}}
\end{fulllineitems}

\index{objects (webapp.models.Territory attribute)}

\begin{fulllineitems}
\phantomsection\label{modules/webapp:webapp.models.Territory.objects}\pysigline{\code{Territory.}\bfcode{objects}\strong{ = \textless{}django.db.models.manager.Manager object\textgreater{}}}
\end{fulllineitems}

\index{rate\_plan (webapp.models.Territory attribute)}

\begin{fulllineitems}
\phantomsection\label{modules/webapp:webapp.models.Territory.rate_plan}\pysigline{\code{Territory.}\bfcode{rate\_plan}}
\end{fulllineitems}


\end{fulllineitems}

\index{Tier (class in webapp.models)}

\begin{fulllineitems}
\phantomsection\label{modules/webapp:webapp.models.Tier}\pysiglinewithargsret{\strong{class }\code{webapp.models.}\bfcode{Tier}}{\emph{id}, \emph{rate\_plan\_id}, \emph{tier\_level}, \emph{max\_percentage\_of\_baseline}, \emph{rate}, \emph{chart\_color}}{}
Bases: \code{django.db.models.base.Model}
\index{Tier.DoesNotExist}

\begin{fulllineitems}
\phantomsection\label{modules/webapp:webapp.models.Tier.DoesNotExist}\pysigline{\strong{exception }\bfcode{DoesNotExist}}
Bases: \code{django.core.exceptions.ObjectDoesNotExist}

\end{fulllineitems}

\index{Tier.MultipleObjectsReturned}

\begin{fulllineitems}
\phantomsection\label{modules/webapp:webapp.models.Tier.MultipleObjectsReturned}\pysigline{\strong{exception }\code{Tier.}\bfcode{MultipleObjectsReturned}}
Bases: \code{django.core.exceptions.MultipleObjectsReturned}

\end{fulllineitems}

\index{devicewebsettings\_set (webapp.models.Tier attribute)}

\begin{fulllineitems}
\phantomsection\label{modules/webapp:webapp.models.Tier.devicewebsettings_set}\pysigline{\code{Tier.}\bfcode{devicewebsettings\_set}}
\end{fulllineitems}

\index{objects (webapp.models.Tier attribute)}

\begin{fulllineitems}
\phantomsection\label{modules/webapp:webapp.models.Tier.objects}\pysigline{\code{Tier.}\bfcode{objects}\strong{ = \textless{}django.db.models.manager.Manager object\textgreater{}}}
\end{fulllineitems}

\index{rate\_plan (webapp.models.Tier attribute)}

\begin{fulllineitems}
\phantomsection\label{modules/webapp:webapp.models.Tier.rate_plan}\pysigline{\code{Tier.}\bfcode{rate\_plan}}
\end{fulllineitems}


\end{fulllineitems}

\index{UserSettings (class in webapp.models)}

\begin{fulllineitems}
\phantomsection\label{modules/webapp:webapp.models.UserSettings}\pysiglinewithargsret{\strong{class }\code{webapp.models.}\bfcode{UserSettings}}{\emph{id}, \emph{user\_id}}{}
Bases: \code{django.db.models.base.Model}
\index{UserSettings.DoesNotExist}

\begin{fulllineitems}
\phantomsection\label{modules/webapp:webapp.models.UserSettings.DoesNotExist}\pysigline{\strong{exception }\bfcode{DoesNotExist}}
Bases: \code{django.core.exceptions.ObjectDoesNotExist}

\end{fulllineitems}

\index{UserSettings.MultipleObjectsReturned}

\begin{fulllineitems}
\phantomsection\label{modules/webapp:webapp.models.UserSettings.MultipleObjectsReturned}\pysigline{\strong{exception }\code{UserSettings.}\bfcode{MultipleObjectsReturned}}
Bases: \code{django.core.exceptions.MultipleObjectsReturned}

\end{fulllineitems}

\index{event\_notification (webapp.models.UserSettings attribute)}

\begin{fulllineitems}
\phantomsection\label{modules/webapp:webapp.models.UserSettings.event_notification}\pysigline{\code{UserSettings.}\bfcode{event\_notification}}
\end{fulllineitems}

\index{interval\_notification (webapp.models.UserSettings attribute)}

\begin{fulllineitems}
\phantomsection\label{modules/webapp:webapp.models.UserSettings.interval_notification}\pysigline{\code{UserSettings.}\bfcode{interval\_notification}}
\end{fulllineitems}

\index{objects (webapp.models.UserSettings attribute)}

\begin{fulllineitems}
\phantomsection\label{modules/webapp:webapp.models.UserSettings.objects}\pysigline{\code{UserSettings.}\bfcode{objects}\strong{ = \textless{}django.db.models.manager.Manager object\textgreater{}}}
\end{fulllineitems}

\index{user (webapp.models.UserSettings attribute)}

\begin{fulllineitems}
\phantomsection\label{modules/webapp:webapp.models.UserSettings.user}\pysigline{\code{UserSettings.}\bfcode{user}}
\end{fulllineitems}


\end{fulllineitems}

\index{UtilityCompany (class in webapp.models)}

\begin{fulllineitems}
\phantomsection\label{modules/webapp:webapp.models.UtilityCompany}\pysiglinewithargsret{\strong{class }\code{webapp.models.}\bfcode{UtilityCompany}}{\emph{id}, \emph{description}}{}
Bases: \code{django.db.models.base.Model}
\index{UtilityCompany.DoesNotExist}

\begin{fulllineitems}
\phantomsection\label{modules/webapp:webapp.models.UtilityCompany.DoesNotExist}\pysigline{\strong{exception }\bfcode{DoesNotExist}}
Bases: \code{django.core.exceptions.ObjectDoesNotExist}

\end{fulllineitems}

\index{UtilityCompany.MultipleObjectsReturned}

\begin{fulllineitems}
\phantomsection\label{modules/webapp:webapp.models.UtilityCompany.MultipleObjectsReturned}\pysigline{\strong{exception }\code{UtilityCompany.}\bfcode{MultipleObjectsReturned}}
Bases: \code{django.core.exceptions.MultipleObjectsReturned}

\end{fulllineitems}

\index{devicewebsettings\_set (webapp.models.UtilityCompany attribute)}

\begin{fulllineitems}
\phantomsection\label{modules/webapp:webapp.models.UtilityCompany.devicewebsettings_set}\pysigline{\code{UtilityCompany.}\bfcode{devicewebsettings\_set}}
\end{fulllineitems}

\index{objects (webapp.models.UtilityCompany attribute)}

\begin{fulllineitems}
\phantomsection\label{modules/webapp:webapp.models.UtilityCompany.objects}\pysigline{\code{UtilityCompany.}\bfcode{objects}\strong{ = \textless{}django.db.models.manager.Manager object\textgreater{}}}
\end{fulllineitems}

\index{rateplan\_set (webapp.models.UtilityCompany attribute)}

\begin{fulllineitems}
\phantomsection\label{modules/webapp:webapp.models.UtilityCompany.rateplan_set}\pysigline{\code{UtilityCompany.}\bfcode{rateplan\_set}}
\end{fulllineitems}


\end{fulllineitems}



\subsection{webapp.tests module}
\label{modules/webapp:module-webapp.tests}\label{modules/webapp:webapp-tests-module}\index{webapp.tests (module)}

\subsection{webapp.timeseries module}
\label{modules/webapp:module-webapp.timeseries}\label{modules/webapp:webapp-timeseries-module}\index{webapp.timeseries (module)}\index{smooth() (in module webapp.timeseries)}

\begin{fulllineitems}
\phantomsection\label{modules/webapp:webapp.timeseries.smooth}\pysiglinewithargsret{\code{webapp.timeseries.}\bfcode{smooth}}{\emph{x}, \emph{window\_len=11}, \emph{window='hanning'}}{}
smooth the data using a window with requested size.

This method is based on the convolution of a scaled window with the signal.
The signal is prepared by introducing reflected copies of the signal 
(with the window size) in both ends so that transient parts are minimized
in the begining and end part of the output signal.
\begin{description}
\item[{input:}] \leavevmode
x: the input signal 
window\_len: the dimension of the smoothing window; should be an odd integer
window: the type of window from `flat', `hanning', `hamming', `bartlett', `blackman'
\begin{quote}

flat window will produce a moving average smoothing.
\end{quote}

\item[{output:}] \leavevmode
the smoothed signal

\end{description}

example:

t=linspace(-2,2,0.1)
x=sin(t)+randn(len(t))*0.1
y=smooth(x)

see also:

numpy.hanning, numpy.hamming, numpy.bartlett, numpy.blackman, numpy.convolve
scipy.signal.lfilter

TODO: the window parameter could be the window itself if an array instead of a string
NOTE: length(output) != length(input), to correct this: return y{[}(window\_len/2-1):-(window\_len/2){]} instead of just y.

\href{http://wiki.scipy.org/Cookbook/SignalSmooth}{http://wiki.scipy.org/Cookbook/SignalSmooth}

\end{fulllineitems}



\subsection{webapp.views module}
\label{modules/webapp:webapp-views-module}\label{modules/webapp:module-webapp.views}\index{webapp.views (module)}\index{Object (class in webapp.views)}

\begin{fulllineitems}
\phantomsection\label{modules/webapp:webapp.views.Object}\pysiglinewithargsret{\strong{class }\code{webapp.views.}\bfcode{Object}}{\emph{serial}}{}
\end{fulllineitems}

\index{SettingsForm (class in webapp.views)}

\begin{fulllineitems}
\phantomsection\label{modules/webapp:webapp.views.SettingsForm}\pysiglinewithargsret{\strong{class }\code{webapp.views.}\bfcode{SettingsForm}}{\emph{*args}, \emph{**kwargs}}{}
Bases: \code{django.forms.forms.Form}
\index{base\_fields (webapp.views.SettingsForm attribute)}

\begin{fulllineitems}
\phantomsection\label{modules/webapp:webapp.views.SettingsForm.base_fields}\pysigline{\bfcode{base\_fields}\strong{ = \{`new\_username': \textless{}django.forms.fields.CharField object at 0x2b6d5a1987d0\textgreater{}, `password1': \textless{}django.forms.fields.CharField object at 0x2b6d5a1a5650\textgreater{}, `password2': \textless{}django.forms.fields.CharField object at 0x2b6d5a1a56d0\textgreater{}, `first\_name': \textless{}django.forms.fields.CharField object at 0x2b6d5a1a5750\textgreater{}, `last\_name': \textless{}django.forms.fields.CharField object at 0x2b6d5a1a5850\textgreater{}, `email': \textless{}django.forms.fields.EmailField object at 0x2b6d5a1a5950\textgreater{}, `notifications': \textless{}django.forms.fields.ChoiceField object at 0x2b6d5a1a59d0\textgreater{}, `share\_with': \textless{}django.forms.fields.ChoiceField object at 0x2b6d5a1a5a90\textgreater{}, `new\_name': \textless{}django.forms.fields.CharField object at 0x2b6d5a1a5b10\textgreater{}, `channel\_1': \textless{}django.forms.fields.ChoiceField object at 0x2b6d5a1a5c10\textgreater{}, `channel\_2': \textless{}django.forms.fields.ChoiceField object at 0x2b6d5a1a5c90\textgreater{}, `channel\_3': \textless{}django.forms.fields.ChoiceField object at 0x2b6d5a1a5d10\textgreater{}, `utility\_companies': \textless{}django.forms.fields.ChoiceField object at 0x2b6d5a1a5d90\textgreater{}, `rate\_plans': \textless{}django.forms.fields.ChoiceField object at 0x2b6d5a1a5e10\textgreater{}, `territories': \textless{}django.forms.fields.ChoiceField object at 0x2b6d5a1a5e90\textgreater{}, `stack': \textless{}django.forms.fields.BooleanField object at 0x2b6d5a1a5ed0\textgreater{}\}}}
\end{fulllineitems}

\index{channel\_1\_choices (webapp.views.SettingsForm attribute)}

\begin{fulllineitems}
\phantomsection\label{modules/webapp:webapp.views.SettingsForm.channel_1_choices}\pysigline{\bfcode{channel\_1\_choices}\strong{ = {[}{]}}}
\end{fulllineitems}

\index{channel\_2\_choices (webapp.views.SettingsForm attribute)}

\begin{fulllineitems}
\phantomsection\label{modules/webapp:webapp.views.SettingsForm.channel_2_choices}\pysigline{\bfcode{channel\_2\_choices}\strong{ = {[}{]}}}
\end{fulllineitems}

\index{channel\_3\_choices (webapp.views.SettingsForm attribute)}

\begin{fulllineitems}
\phantomsection\label{modules/webapp:webapp.views.SettingsForm.channel_3_choices}\pysigline{\bfcode{channel\_3\_choices}\strong{ = {[}{]}}}
\end{fulllineitems}

\index{clean\_password2() (webapp.views.SettingsForm method)}

\begin{fulllineitems}
\phantomsection\label{modules/webapp:webapp.views.SettingsForm.clean_password2}\pysiglinewithargsret{\bfcode{clean\_password2}}{}{}
\end{fulllineitems}

\index{error\_messages (webapp.views.SettingsForm attribute)}

\begin{fulllineitems}
\phantomsection\label{modules/webapp:webapp.views.SettingsForm.error_messages}\pysigline{\bfcode{error\_messages}\strong{ = \{`password\_mismatch': ``The two password fields didn't match.''\}}}
\end{fulllineitems}

\index{get\_notifications() (webapp.views.SettingsForm method)}

\begin{fulllineitems}
\phantomsection\label{modules/webapp:webapp.views.SettingsForm.get_notifications}\pysiglinewithargsret{\bfcode{get\_notifications}}{\emph{*args}, \emph{**kwargs}}{}
\end{fulllineitems}

\index{get\_rate\_plans() (webapp.views.SettingsForm method)}

\begin{fulllineitems}
\phantomsection\label{modules/webapp:webapp.views.SettingsForm.get_rate_plans}\pysiglinewithargsret{\bfcode{get\_rate\_plans}}{\emph{*args}, \emph{**kwargs}}{}
\end{fulllineitems}

\index{get\_territories() (webapp.views.SettingsForm method)}

\begin{fulllineitems}
\phantomsection\label{modules/webapp:webapp.views.SettingsForm.get_territories}\pysiglinewithargsret{\bfcode{get\_territories}}{\emph{*args}, \emph{**kwargs}}{}
\end{fulllineitems}

\index{get\_utility\_companies() (webapp.views.SettingsForm method)}

\begin{fulllineitems}
\phantomsection\label{modules/webapp:webapp.views.SettingsForm.get_utility_companies}\pysiglinewithargsret{\bfcode{get\_utility\_companies}}{\emph{*args}, \emph{**kwargs}}{}
\end{fulllineitems}

\index{media (webapp.views.SettingsForm attribute)}

\begin{fulllineitems}
\phantomsection\label{modules/webapp:webapp.views.SettingsForm.media}\pysigline{\bfcode{media}}
\end{fulllineitems}

\index{notification\_choices (webapp.views.SettingsForm attribute)}

\begin{fulllineitems}
\phantomsection\label{modules/webapp:webapp.views.SettingsForm.notification_choices}\pysigline{\bfcode{notification\_choices}\strong{ = {[}{]}}}
\end{fulllineitems}

\index{rate\_plan\_choices (webapp.views.SettingsForm attribute)}

\begin{fulllineitems}
\phantomsection\label{modules/webapp:webapp.views.SettingsForm.rate_plan_choices}\pysigline{\bfcode{rate\_plan\_choices}\strong{ = {[}{]}}}
\end{fulllineitems}

\index{share\_with\_choices (webapp.views.SettingsForm attribute)}

\begin{fulllineitems}
\phantomsection\label{modules/webapp:webapp.views.SettingsForm.share_with_choices}\pysigline{\bfcode{share\_with\_choices}\strong{ = {[}{]}}}
\end{fulllineitems}

\index{territory\_choices (webapp.views.SettingsForm attribute)}

\begin{fulllineitems}
\phantomsection\label{modules/webapp:webapp.views.SettingsForm.territory_choices}\pysigline{\bfcode{territory\_choices}\strong{ = {[}{]}}}
\end{fulllineitems}

\index{utility\_company\_choices (webapp.views.SettingsForm attribute)}

\begin{fulllineitems}
\phantomsection\label{modules/webapp:webapp.views.SettingsForm.utility_company_choices}\pysigline{\bfcode{utility\_company\_choices}\strong{ = {[}{]}}}
\end{fulllineitems}


\end{fulllineitems}

\index{billing\_information() (in module webapp.views)}

\begin{fulllineitems}
\phantomsection\label{modules/webapp:webapp.views.billing_information}\pysiglinewithargsret{\code{webapp.views.}\bfcode{billing\_information}}{\emph{request}, \emph{*args}, \emph{**kwargs}}{}
\end{fulllineitems}

\index{chartify() (in module webapp.views)}

\begin{fulllineitems}
\phantomsection\label{modules/webapp:webapp.views.chartify}\pysiglinewithargsret{\code{webapp.views.}\bfcode{chartify}}{\emph{data}}{}
\end{fulllineitems}

\index{charts\_deprecated() (in module webapp.views)}

\begin{fulllineitems}
\phantomsection\label{modules/webapp:webapp.views.charts_deprecated}\pysiglinewithargsret{\code{webapp.views.}\bfcode{charts\_deprecated}}{\emph{request}, \emph{*args}, \emph{**kwargs}}{}
\end{fulllineitems}

\index{circuits\_information() (in module webapp.views)}

\begin{fulllineitems}
\phantomsection\label{modules/webapp:webapp.views.circuits_information}\pysiglinewithargsret{\code{webapp.views.}\bfcode{circuits\_information}}{\emph{request}, \emph{*args}, \emph{**kwargs}}{}
\end{fulllineitems}

\index{dashboard() (in module webapp.views)}

\begin{fulllineitems}
\phantomsection\label{modules/webapp:webapp.views.dashboard}\pysiglinewithargsret{\code{webapp.views.}\bfcode{dashboard}}{\emph{request}, \emph{*args}, \emph{**kwargs}}{}
\end{fulllineitems}

\index{dashboard\_update() (in module webapp.views)}

\begin{fulllineitems}
\phantomsection\label{modules/webapp:webapp.views.dashboard_update}\pysiglinewithargsret{\code{webapp.views.}\bfcode{dashboard\_update}}{\emph{request}, \emph{*args}, \emph{**kwargs}}{}
\end{fulllineitems}

\index{default\_chart() (in module webapp.views)}

\begin{fulllineitems}
\phantomsection\label{modules/webapp:webapp.views.default_chart}\pysiglinewithargsret{\code{webapp.views.}\bfcode{default\_chart}}{\emph{request}, \emph{*args}, \emph{**kwargs}}{}
\end{fulllineitems}

\index{device\_chart() (in module webapp.views)}

\begin{fulllineitems}
\phantomsection\label{modules/webapp:webapp.views.device_chart}\pysiglinewithargsret{\code{webapp.views.}\bfcode{device\_chart}}{\emph{request}, \emph{*args}, \emph{**kwargs}}{}
\end{fulllineitems}

\index{device\_data() (in module webapp.views)}

\begin{fulllineitems}
\phantomsection\label{modules/webapp:webapp.views.device_data}\pysiglinewithargsret{\code{webapp.views.}\bfcode{device\_data}}{\emph{request}, \emph{*args}, \emph{**kwargs}}{}
\end{fulllineitems}

\index{device\_is\_online() (in module webapp.views)}

\begin{fulllineitems}
\phantomsection\label{modules/webapp:webapp.views.device_is_online}\pysiglinewithargsret{\code{webapp.views.}\bfcode{device\_is\_online}}{\emph{device}}{}
\end{fulllineitems}

\index{device\_location() (in module webapp.views)}

\begin{fulllineitems}
\phantomsection\label{modules/webapp:webapp.views.device_location}\pysiglinewithargsret{\code{webapp.views.}\bfcode{device\_location}}{\emph{request}, \emph{*args}, \emph{**kwargs}}{}
\end{fulllineitems}

\index{device\_status() (in module webapp.views)}

\begin{fulllineitems}
\phantomsection\label{modules/webapp:webapp.views.device_status}\pysiglinewithargsret{\code{webapp.views.}\bfcode{device\_status}}{\emph{request}, \emph{*args}, \emph{**kwargs}}{}
\end{fulllineitems}

\index{export\_data() (in module webapp.views)}

\begin{fulllineitems}
\phantomsection\label{modules/webapp:webapp.views.export_data}\pysiglinewithargsret{\code{webapp.views.}\bfcode{export\_data}}{\emph{request}, \emph{*args}, \emph{**kwargs}}{}
\end{fulllineitems}

\index{generate\_average\_wattage\_usage() (in module webapp.views)}

\begin{fulllineitems}
\phantomsection\label{modules/webapp:webapp.views.generate_average_wattage_usage}\pysiglinewithargsret{\code{webapp.views.}\bfcode{generate\_average\_wattage\_usage}}{\emph{request}, \emph{*args}, \emph{**kwargs}}{}
\end{fulllineitems}

\index{generate\_heatmap\_data() (in module webapp.views)}

\begin{fulllineitems}
\phantomsection\label{modules/webapp:webapp.views.generate_heatmap_data}\pysiglinewithargsret{\code{webapp.views.}\bfcode{generate\_heatmap\_data}}{\emph{serial}}{}
\end{fulllineitems}

\index{get\_wattage\_usage() (in module webapp.views)}

\begin{fulllineitems}
\phantomsection\label{modules/webapp:webapp.views.get_wattage_usage}\pysiglinewithargsret{\code{webapp.views.}\bfcode{get\_wattage\_usage}}{\emph{request}, \emph{*args}, \emph{**kwargs}}{}
\end{fulllineitems}

\index{group\_by\_mean() (in module webapp.views)}

\begin{fulllineitems}
\phantomsection\label{modules/webapp:webapp.views.group_by_mean}\pysiglinewithargsret{\code{webapp.views.}\bfcode{group\_by\_mean}}{\emph{serial}, \emph{unit}, \emph{start}, \emph{stop}, \emph{localtime}, \emph{circuit\_pk}}{}
\end{fulllineitems}

\index{heatmap() (in module webapp.views)}

\begin{fulllineitems}
\phantomsection\label{modules/webapp:webapp.views.heatmap}\pysiglinewithargsret{\code{webapp.views.}\bfcode{heatmap}}{\emph{request}, \emph{*args}, \emph{**kwargs}}{}
\end{fulllineitems}

\index{landing() (in module webapp.views)}

\begin{fulllineitems}
\phantomsection\label{modules/webapp:webapp.views.landing}\pysiglinewithargsret{\code{webapp.views.}\bfcode{landing}}{\emph{request}, \emph{*args}, \emph{**kwargs}}{}
\end{fulllineitems}

\index{make\_choices() (in module webapp.views)}

\begin{fulllineitems}
\phantomsection\label{modules/webapp:webapp.views.make_choices}\pysiglinewithargsret{\code{webapp.views.}\bfcode{make\_choices}}{\emph{querysets}}{}
\end{fulllineitems}

\index{merge\_subs() (in module webapp.views)}

\begin{fulllineitems}
\phantomsection\label{modules/webapp:webapp.views.merge_subs}\pysiglinewithargsret{\code{webapp.views.}\bfcode{merge\_subs}}{\emph{lst\_of\_lsts}}{}
\end{fulllineitems}

\index{remove\_device() (in module webapp.views)}

\begin{fulllineitems}
\phantomsection\label{modules/webapp:webapp.views.remove_device}\pysiglinewithargsret{\code{webapp.views.}\bfcode{remove\_device}}{\emph{request}, \emph{*args}, \emph{**kwargs}}{}
\end{fulllineitems}

\index{settings() (in module webapp.views)}

\begin{fulllineitems}
\phantomsection\label{modules/webapp:webapp.views.settings}\pysiglinewithargsret{\code{webapp.views.}\bfcode{settings}}{\emph{*args}, \emph{**kwargs}}{}
\end{fulllineitems}

\index{settings\_account() (in module webapp.views)}

\begin{fulllineitems}
\phantomsection\label{modules/webapp:webapp.views.settings_account}\pysiglinewithargsret{\code{webapp.views.}\bfcode{settings\_account}}{\emph{request}, \emph{*args}, \emph{**kwargs}}{}
\end{fulllineitems}

\index{settings\_change\_device() (in module webapp.views)}

\begin{fulllineitems}
\phantomsection\label{modules/webapp:webapp.views.settings_change_device}\pysiglinewithargsret{\code{webapp.views.}\bfcode{settings\_change\_device}}{\emph{request}, \emph{*args}, \emph{**kwargs}}{}
\end{fulllineitems}

\index{settings\_dashboard() (in module webapp.views)}

\begin{fulllineitems}
\phantomsection\label{modules/webapp:webapp.views.settings_dashboard}\pysiglinewithargsret{\code{webapp.views.}\bfcode{settings\_dashboard}}{\emph{request}, \emph{*args}, \emph{**kwargs}}{}
\end{fulllineitems}

\index{settings\_device() (in module webapp.views)}

\begin{fulllineitems}
\phantomsection\label{modules/webapp:webapp.views.settings_device}\pysiglinewithargsret{\code{webapp.views.}\bfcode{settings\_device}}{\emph{request}, \emph{*args}, \emph{**kwargs}}{}
\end{fulllineitems}



\subsection{Module contents}
\label{modules/webapp:module-webapp}\label{modules/webapp:module-contents}\index{webapp (module)}

\section{farmer package}
\label{modules/farmer:farmer-package}\label{modules/farmer::doc}

\subsection{Submodules}
\label{modules/farmer:submodules}

\subsection{farmer.admin module}
\label{modules/farmer:module-farmer.admin}\label{modules/farmer:farmer-admin-module}\index{farmer.admin (module)}\index{DeviceSettingsAdmin (class in farmer.admin)}

\begin{fulllineitems}
\phantomsection\label{modules/farmer:farmer.admin.DeviceSettingsAdmin}\pysiglinewithargsret{\strong{class }\code{farmer.admin.}\bfcode{DeviceSettingsAdmin}}{\emph{model}, \emph{admin\_site}}{}
Bases: \code{django.contrib.admin.options.ModelAdmin}
\index{media (farmer.admin.DeviceSettingsAdmin attribute)}

\begin{fulllineitems}
\phantomsection\label{modules/farmer:farmer.admin.DeviceSettingsAdmin.media}\pysigline{\bfcode{media}}
\end{fulllineitems}

\index{verbose\_name\_plural (farmer.admin.DeviceSettingsAdmin attribute)}

\begin{fulllineitems}
\phantomsection\label{modules/farmer:farmer.admin.DeviceSettingsAdmin.verbose_name_plural}\pysigline{\bfcode{verbose\_name\_plural}\strong{ = `devicesettings'}}
\end{fulllineitems}


\end{fulllineitems}

\index{DeviceSettingsInline (class in farmer.admin)}

\begin{fulllineitems}
\phantomsection\label{modules/farmer:farmer.admin.DeviceSettingsInline}\pysiglinewithargsret{\strong{class }\code{farmer.admin.}\bfcode{DeviceSettingsInline}}{\emph{parent\_model}, \emph{admin\_site}}{}
Bases: \code{django.contrib.admin.options.StackedInline}
\index{can\_delete (farmer.admin.DeviceSettingsInline attribute)}

\begin{fulllineitems}
\phantomsection\label{modules/farmer:farmer.admin.DeviceSettingsInline.can_delete}\pysigline{\bfcode{can\_delete}\strong{ = False}}
\end{fulllineitems}

\index{media (farmer.admin.DeviceSettingsInline attribute)}

\begin{fulllineitems}
\phantomsection\label{modules/farmer:farmer.admin.DeviceSettingsInline.media}\pysigline{\bfcode{media}}
\end{fulllineitems}

\index{model (farmer.admin.DeviceSettingsInline attribute)}

\begin{fulllineitems}
\phantomsection\label{modules/farmer:farmer.admin.DeviceSettingsInline.model}\pysigline{\bfcode{model}}
alias of \code{DeviceSettings}

\end{fulllineitems}


\end{fulllineitems}



\subsection{farmer.models module}
\label{modules/farmer:module-farmer.models}\label{modules/farmer:farmer-models-module}\index{farmer.models (module)}\index{DeviceSettings (class in farmer.models)}

\begin{fulllineitems}
\phantomsection\label{modules/farmer:farmer.models.DeviceSettings}\pysiglinewithargsret{\strong{class }\code{farmer.models.}\bfcode{DeviceSettings}}{\emph{device\_id}, \emph{device\_serial}, \emph{main\_channel}, \emph{adc\_sample\_rate}, \emph{transmission\_rate\_milliseconds}, \emph{date\_now}}{}
Bases: \code{django.db.models.base.Model}
\index{CHANNEL\_CHOICES (farmer.models.DeviceSettings attribute)}

\begin{fulllineitems}
\phantomsection\label{modules/farmer:farmer.models.DeviceSettings.CHANNEL_CHOICES}\pysigline{\bfcode{CHANNEL\_CHOICES}\strong{ = ((1, `Channel 1'), (2, `Channel 2'), (3, `Channel 3'), (4, `Channel 4'))}}
\end{fulllineitems}

\index{DeviceSettings.DoesNotExist}

\begin{fulllineitems}
\phantomsection\label{modules/farmer:farmer.models.DeviceSettings.DoesNotExist}\pysigline{\strong{exception }\bfcode{DoesNotExist}}
Bases: \code{django.core.exceptions.ObjectDoesNotExist}

\end{fulllineitems}

\index{DeviceSettings.MultipleObjectsReturned}

\begin{fulllineitems}
\phantomsection\label{modules/farmer:farmer.models.DeviceSettings.MultipleObjectsReturned}\pysigline{\strong{exception }\code{DeviceSettings.}\bfcode{MultipleObjectsReturned}}
Bases: \code{django.core.exceptions.MultipleObjectsReturned}

\end{fulllineitems}

\index{SAMPLE\_RATE\_CHOICES (farmer.models.DeviceSettings attribute)}

\begin{fulllineitems}
\phantomsection\label{modules/farmer:farmer.models.DeviceSettings.SAMPLE_RATE_CHOICES}\pysigline{\code{DeviceSettings.}\bfcode{SAMPLE\_RATE\_CHOICES}\strong{ = ((0, 125000), (1, 62500), (2, 31250), (3, 15625), (4, 7812.5), (5, 3906.25), (6, 1953.125), (7, 976.5625))}}
\end{fulllineitems}

\index{device (farmer.models.DeviceSettings attribute)}

\begin{fulllineitems}
\phantomsection\label{modules/farmer:farmer.models.DeviceSettings.device}\pysigline{\code{DeviceSettings.}\bfcode{device}}
\end{fulllineitems}

\index{get\_adc\_sample\_rate\_display() (farmer.models.DeviceSettings method)}

\begin{fulllineitems}
\phantomsection\label{modules/farmer:farmer.models.DeviceSettings.get_adc_sample_rate_display}\pysiglinewithargsret{\code{DeviceSettings.}\bfcode{get\_adc\_sample\_rate\_display}}{\emph{*moreargs}, \emph{**morekwargs}}{}
\end{fulllineitems}

\index{get\_main\_channel\_display() (farmer.models.DeviceSettings method)}

\begin{fulllineitems}
\phantomsection\label{modules/farmer:farmer.models.DeviceSettings.get_main_channel_display}\pysiglinewithargsret{\code{DeviceSettings.}\bfcode{get\_main\_channel\_display}}{\emph{*moreargs}, \emph{**morekwargs}}{}
\end{fulllineitems}

\index{objects (farmer.models.DeviceSettings attribute)}

\begin{fulllineitems}
\phantomsection\label{modules/farmer:farmer.models.DeviceSettings.objects}\pysigline{\code{DeviceSettings.}\bfcode{objects}\strong{ = \textless{}django.db.models.manager.Manager object\textgreater{}}}
\end{fulllineitems}

\index{save() (farmer.models.DeviceSettings method)}

\begin{fulllineitems}
\phantomsection\label{modules/farmer:farmer.models.DeviceSettings.save}\pysiglinewithargsret{\code{DeviceSettings.}\bfcode{save}}{\emph{**kwargs}}{}
\end{fulllineitems}


\end{fulllineitems}



\subsection{farmer.serializers module}
\label{modules/farmer:module-farmer.serializers}\label{modules/farmer:farmer-serializers-module}\index{farmer.serializers (module)}\index{DeviceSettingsSerializer (class in farmer.serializers)}

\begin{fulllineitems}
\phantomsection\label{modules/farmer:farmer.serializers.DeviceSettingsSerializer}\pysiglinewithargsret{\strong{class }\code{farmer.serializers.}\bfcode{DeviceSettingsSerializer}}{\emph{instance=None}, \emph{data=\textless{}class rest\_framework.fields.empty\textgreater{}}, \emph{**kwargs}}{}
Bases: \code{rest\_framework.serializers.ModelSerializer}
\index{DeviceSettingsSerializer.Meta (class in farmer.serializers)}

\begin{fulllineitems}
\phantomsection\label{modules/farmer:farmer.serializers.DeviceSettingsSerializer.Meta}\pysigline{\strong{class }\bfcode{Meta}}~\index{fields (farmer.serializers.DeviceSettingsSerializer.Meta attribute)}

\begin{fulllineitems}
\phantomsection\label{modules/farmer:farmer.serializers.DeviceSettingsSerializer.Meta.fields}\pysigline{\bfcode{fields}\strong{ = (`device', `device\_serial', `main\_channel', `transmission\_rate\_milliseconds', `adc\_sample\_rate', `date\_now')}}
\end{fulllineitems}

\index{model (farmer.serializers.DeviceSettingsSerializer.Meta attribute)}

\begin{fulllineitems}
\phantomsection\label{modules/farmer:farmer.serializers.DeviceSettingsSerializer.Meta.model}\pysigline{\bfcode{model}}
alias of \code{DeviceSettings}

\end{fulllineitems}


\end{fulllineitems}


\end{fulllineitems}



\subsection{farmer.tests module}
\label{modules/farmer:farmer-tests-module}\label{modules/farmer:module-farmer.tests}\index{farmer.tests (module)}

\subsection{farmer.views module}
\label{modules/farmer:farmer-views-module}\label{modules/farmer:module-farmer.views}\index{farmer.views (module)}\index{DeviceSettingsViewSet (class in farmer.views)}

\begin{fulllineitems}
\phantomsection\label{modules/farmer:farmer.views.DeviceSettingsViewSet}\pysiglinewithargsret{\strong{class }\code{farmer.views.}\bfcode{DeviceSettingsViewSet}}{\emph{**kwargs}}{}
Bases: \code{rest\_framework.viewsets.ModelViewSet}

API endpoint that allows devicesettings to be viewed or edited.
\index{list() (farmer.views.DeviceSettingsViewSet method)}

\begin{fulllineitems}
\phantomsection\label{modules/farmer:farmer.views.DeviceSettingsViewSet.list}\pysiglinewithargsret{\bfcode{list}}{\emph{request}}{}
\end{fulllineitems}

\index{queryset (farmer.views.DeviceSettingsViewSet attribute)}

\begin{fulllineitems}
\phantomsection\label{modules/farmer:farmer.views.DeviceSettingsViewSet.queryset}\pysigline{\bfcode{queryset}}
\end{fulllineitems}

\index{retrieve() (farmer.views.DeviceSettingsViewSet method)}

\begin{fulllineitems}
\phantomsection\label{modules/farmer:farmer.views.DeviceSettingsViewSet.retrieve}\pysiglinewithargsret{\bfcode{retrieve}}{\emph{request}, \emph{pk=None}}{}
\end{fulllineitems}

\index{serializer\_class (farmer.views.DeviceSettingsViewSet attribute)}

\begin{fulllineitems}
\phantomsection\label{modules/farmer:farmer.views.DeviceSettingsViewSet.serializer_class}\pysigline{\bfcode{serializer\_class}}
alias of \code{DeviceSettingsSerializer}

\end{fulllineitems}


\end{fulllineitems}



\subsection{Module contents}
\label{modules/farmer:module-contents}\label{modules/farmer:module-farmer}\index{farmer (module)}

\chapter{Indices and tables}
\label{index:indices-and-tables}\begin{itemize}
\item {} 
\DUspan{xref,std,std-ref}{genindex}

\item {} 
\DUspan{xref,std,std-ref}{modindex}

\item {} 
\DUspan{xref,std,std-ref}{search}

\end{itemize}


\renewcommand{\indexname}{Python Module Index}
\begin{theindex}
\def\bigletter#1{{\Large\sffamily#1}\nopagebreak\vspace{1mm}}
\bigletter{f}
\item {\texttt{farmer}}, \pageref{modules/farmer:module-farmer}
\item {\texttt{farmer.admin}}, \pageref{modules/farmer:module-farmer.admin}
\item {\texttt{farmer.models}}, \pageref{modules/farmer:module-farmer.models}
\item {\texttt{farmer.serializers}}, \pageref{modules/farmer:module-farmer.serializers}
\item {\texttt{farmer.tests}}, \pageref{modules/farmer:module-farmer.tests}
\item {\texttt{farmer.views}}, \pageref{modules/farmer:module-farmer.views}
\indexspace
\bigletter{m}
\item {\texttt{microdata}}, \pageref{modules/microdata:module-microdata}
\item {\texttt{microdata.admin}}, \pageref{modules/microdata:module-microdata.admin}
\item {\texttt{microdata.management}}, \pageref{modules/microdata.management:module-microdata.management}
\item {\texttt{microdata.management.commands}}, \pageref{modules/microdata.management.commands:module-microdata.management.commands}
\item {\texttt{microdata.management.commands.archive\_database}}, \pageref{modules/microdata.management.commands:module-microdata.management.commands.archive_database}
\item {\texttt{microdata.management.commands.check\_glacier\_jobs}}, \pageref{modules/microdata.management.commands:module-microdata.management.commands.check_glacier_jobs}
\item {\texttt{microdata.models}}, \pageref{modules/microdata:module-microdata.models}
\item {\texttt{microdata.serializers}}, \pageref{modules/microdata:module-microdata.serializers}
\item {\texttt{microdata.tests}}, \pageref{modules/microdata:module-microdata.tests}
\item {\texttt{microdata.views}}, \pageref{modules/microdata:module-microdata.views}
\indexspace
\bigletter{w}
\item {\texttt{webapp}}, \pageref{modules/webapp:module-webapp}
\item {\texttt{webapp.admin}}, \pageref{modules/webapp:module-webapp.admin}
\item {\texttt{webapp.device\_dictionary}}, \pageref{modules/webapp:module-webapp.device_dictionary}
\item {\texttt{webapp.management}}, \pageref{modules/webapp.management:module-webapp.management}
\item {\texttt{webapp.management.commands}}, \pageref{modules/webapp.management.commands:module-webapp.management.commands}
\item {\texttt{webapp.management.commands.reset\_kilowatt\_accumulations}}, \pageref{modules/webapp.management.commands:module-webapp.management.commands.reset_kilowatt_accumulations}
\item {\texttt{webapp.models}}, \pageref{modules/webapp:module-webapp.models}
\item {\texttt{webapp.tests}}, \pageref{modules/webapp:module-webapp.tests}
\item {\texttt{webapp.timeseries}}, \pageref{modules/webapp:module-webapp.timeseries}
\item {\texttt{webapp.views}}, \pageref{modules/webapp:module-webapp.views}
\end{theindex}

\renewcommand{\indexname}{Index}
\printindex
\end{document}
